\section{Pavages du plan}
\subsection{Tuiles}

% def:tuile
\begin{definition}
	Une \emph{tuile} de $\R^2$ est un sous-ensemble $\mathbf{T} \subset \R^2$ qui est
compact (fermé, borné) et d'intérieur $\interior{\mathbf{T}}$ non vide.
\end{definition}

% def:simplexe
\begin{definition}
	Un \emph{simplexe} $\varsigma$ de $\R^2$ est le triangle plein
	determiné par trois points
	distincts non-alignés $\vec{v}_0, \vec{v}_1, \vec{v}_2 \in \R^2$. C'est l'ensemble
	\begin{equation*}
		\varsigma = \overline{\vec{v}_0\vec{v}_1\vec{v}_2} := \{\alpha_0\vec{v}_0
			+ \alpha_1\vec{v}_1 + \alpha_2\vec{v}_2\  |\  \alpha_0, \alpha_1,
			\alpha_2 \in \R^*_+, \alpha_0 + \alpha_1 + \alpha_2 = 1\}
	\end{equation*}
	Les faces de $\varsigma$ (ou encore arêtes) sont les segments $[\vec{v}_i,
	\vec{v}_j], i \neq j$. Les sommets du simplexe sont les points $\vec{v}_0,
	\vec{v}_1, \vec{v}_2$.
\end{definition}

% def:polygone-plein
\begin{definition}
	Un \emph{polygone plein} $\mathbf{P}$ est une reunion finie, non-vide de simplexes
	$\varsigma_1,
	\dots, \varsigma_n \subset \R^2$ tous distincts
	\begin{equation*}
		\mathbf{P} := \bigcup_{i=1}^n\varsigma_i
	\end{equation*}
	Dont l'interieur $\interior{\mathbf{P}}$ est connexe par arcs. Les faces (ou arêtes)
	de $\mathbf{P}$ sont les segments contenus dans $\boundary{\mathbf{P}}$ qui sont
	maximaux pour cette proprieté. Les sommets de $\mathbf{P}$ sont les
	intersections de deux faces distinctes.
\end{definition}

% subsection:pavages
\subsection{Pavages}

% def:pavage
\begin{definition}
	Un \emph{pavage du plan} a une tuile, $\mathcal{P}(\mathbf{T}, \Phi)$ est un
	ensemble de tuiles de $\R^2$ obtenu à partir d'une tuile $\mathbf{T}$
	et d'un sous-ensemble $\Phi \subset \Isom(\R^2)$ tel que
	\begin{equation*}
		\R^2 = \bigcup_{\varphi \in \Phi} \varphi(\mathbf{T})
	\end{equation*}
	et tel que deux tuiles distinctes ne peuvent s'intersecter que suivant leur
	bord, donc p.t. $\varphi, \varphi' \in \Phi$ 
	t.q. $\varphi(\mathbf{T}) \neq \varphi'(\mathbf{T})$
	\begin{equation*}
		\varphi(\mathbf{T})^\circ \cap \varphi(\mathbf{T})^\circ = \emptyset,
	\end{equation*}
	ou de maniere equivalente,
	\begin{equation*}
		\varphi(\mathbf{T}) \cap \varphi'(\mathbf{T}) \subset
		\boundary{\varphi(\mathbf{T})} \cap \boundary{\varphi'(\mathbf{T})}.
	\end{equation*}	
\end{definition}

% prop:tuile-isom
\begin{proposition}
	Soit $\mathbf{T}$ une tuile et $\varphi \in \Isom(\R^2)$ une isometrie alors
	$\varphi(\mathbf{T})$ est une tuile et si $\mathbf{T}$ est polygonale,
	$\varphi(\mathbf{T})$ l'est aussi.
\end{proposition}

% prop:pavage-isom
\begin{proposition}
	Soit $\mathcal{P}(\mathbf{T}, \Phi)$ un pavage et $\psi \in \Isom(\R^2)$ une
	isometrie, alors 
	\begin{equation*}
		\psi(\mathcal{P}(\mathbf{T}, \Phi)) = \mathcal{P}(\mathbf{T}, \psi \circ \Phi)
		= \mathcal{P}(\psi(\mathbf{T}), \Ad(\psi)(\Phi))
	\end{equation*}
	est aussi un pavage.
\end{proposition}

% def:groupe-isom-pavage
\begin{definition}
	Soit $\mathcal{P} = \mathcal{P}(\mathbf{T}, \Phi)$ un pavage, le \emph{groupe
	des isometries fixant} $\mathcal{P}$ est l'ensemble
	\begin{equation*}
		\Isom(\R^2)_\mathcal{P} := 
		\{\psi \in \Isom(\R^2)\ |\ \psi(\mathcal{P}) = \mathcal{P}\}
	\end{equation*}
\end{definition}

% def:pavage-de-reseaux
\subsection{Pavages de Reseaux}

% def:reseau
\begin{definition}
	Un reseau $\Gamma$ est un sous-groupe additif de $(\R^2, +)$
	engendré par deux vecteurs, $\vec{u}, \vec{v} \in \R^2$ linéairement
	indépendants:
	\begin{equation*}
		\Gamma = \Z\vec{v} + \Z\vec{u}
	\end{equation*}
\end{definition}

% def:parallelogramme-fondamental
\begin{definition}
	Soit $\Gamma = \Z\vec{u} + \Z\vec{v}$ un reseau. Le parallélogramme
	fondamental porté par $\{\vec{u}, \vec{v}\}$ est l'ensemble
	\begin{equation*}
		\mathbf{P}_{\vec{u},\vec{v}} :=
		\{\alpha\vec{u} + \beta\vec{v}\ |\ \alpha, \beta \in [0, 1]\}
	\end{equation*}
	Soit $T(\Gamma) = \{t_\gamma\ |\ \gamma \in \Gamma\}$ le groupe des
	translations par les vecteurs de $\Gamma$, alors pour tout point
	$\vec{x}_0 \in \R^2$,
	$\mathcal{P}(\vec{x}_0 + \mathbf{P}_{\vec{u},\vec{v}}, T(\Gamma))$
	est un pavage du plan.
\end{definition}

\begin{remark}
	$\mathcal{P}(\vec{x}_0 + \mathbf{P}_{\vec{u},\vec{v}}, T(\Gamma))$
	c'est un pavage dont la donnée du sous-ensemble de $\Isom(\R^2)$ est un sous
	groupe de $\Isom(\R^2)$.
\end{remark}

% subsection:groupes-cristallographiques
\subsection{Les groupes cristallographiques du plan}

% def:pavage-regulier
\begin{definition}
	Un pavage $\mathcal{P}$ est dit \emph{regulier} si son groupe d'isometries
	$\Isom(\R^2)_\mathcal{P}$ contient au moins deux translations $t_\vec{u}$ et
	$t_\vec{v}$ où $\vec{u}, \vec{v} \in \R^2$ sont linéairement indépendants. En
	d'autres termes, posons $\Gamma = \Z\vec{u} + \Z\vec{v}$, alors
	$T(\Gamma)\subset \Isom(\R^2)_\mathcal{P}$.
\end{definition}

% def:groupe-cristallographique
\begin{definition}
	Le groupe des isometries d'un pavage regulier $\mathcal{P}$,
	$G_\mathcal{P} \subset \Isom(\R^2)$ est dit \emph{cristallographique}.
\end{definition}

\begin{notation}
	Soit $G = G_\mathcal{P}$ un groupe cristallographique, on note
	\begin{equation*}
		G^+=G\cap\Isom(\R^2)^+
	\end{equation*}
	le sous groupe des rotations affines,
	\begin{equation*}
		T_G = G^+ \cap T(\R^2)
	\end{equation*}
	le sous-groupe des translations et
	\begin{equation*}
		\Gamma_G = \{\vec{v} \in \R^2\ |\ t_\vec{v} \in T_G\} \subset \R^2
	\end{equation*}
	le sous-groupe additif des vecteurs correspondant à $T_G$.
	On pourra traiter $\Gamma_G$ comme un sous-groupe du groupe $(\C, +)$.

\end{notation}

% prop:sous-groupes-distingues
\begin{proposition}
	Les sous groupes $T_G$ et $G^+$ sont distingués dans $G$.
\end{proposition}

\begin{proof}
	$T_G = \ker(\lin_{|G})$, où $\lin_{|G}$ est l'application
	\begin{align*}
		\lin: \Isom(\R^2) &\to \Isom(\R^2)_0\\
		\varphi &\mapsto \varphi_0
	\end{align*}
	restreinte à $G$ et donc on a bien $T_G \triangleleft G$.  
	
	De même, $G^+ = \ker((\det \circ \lin)_{|G})$, où $(\det \circ \lin)_{|G}$
	est l'application
	\begin{align*}
		\det \circ \lin: \Isom(\R^2) &\to \R\\
		\phi &\mapsto \det(M_{\phi_0})
	\end{align*}
	restreinte à $G$ et donc on a bien $G^+ \triangleleft G$.
\end{proof}

\subsubsection{Le sous-groupe des translations}

% lem:tuiles-boule
\begin{lemma}
	\label{lem:tuiles-boule}
	Pour toute boule ouverte B = $B(\vec{x}, r)$, le nombre de tuiles
	du pavage $\mathcal{P} = \mathcal{P}(\mathbf{T}, \Psi)$ qui intersectent B est fini.
\end{lemma}
\begin{proof}
	Soit $\vec{x}_0 \in \mathbf{T}^\circ$ et $r_0 > 0$ tel que
	$B(\vec{x}_0, r_0) \subset \mathbf{T}$, et
	posons $d_{max} = \max \{\|\vec{x} - \vec{y}\|:\vec{x}, \vec{y} \in
	\mathbf{T}\}$.
	Soit $\mathbf{T}' \in \mathcal{P}$ une tuile du pavage qui intersecte $B$,
	alors $\mathbf{T}' \subset B(\vec{x}, r + d_{max})$.
	Notons $N$ le nombre de tuiles de $\mathcal{P}$ qui intersectent $B$.
	Alors puisque les tuiles sont d'aire non-nul, on a que
	l'aire de la reunion de toutes les tuiles qui intersectent $B$
	est plus grande ou égale à
	\begin{equation*}
		Aire(B(\vec{x_0}, r_0)) \times N = \pi r_0^2 \times N.
	\end{equation*}
	Cette aire est majorée par l'aire du disque $D(\vec{x}, r + d_{max})$,
	on a donc
	\begin{equation*}
		\pi r_0^2 \times N \leq \pi(r + d_{max})^2
	\end{equation*}
	et ainsi $N < +\infty$
\end{proof}

% lem:translations-boule
\begin{lemma}
	\label{lem:translations-boule}
	Soit $\mathcal{P} = \mathcal{P}(\mathbf{T}, \Psi)$ un pavage régulier.
	Pour toute boule ouverte $B = B(\vec{x}, r)$, l'ensemble
	$\Gamma_G \cap B$ est fini.
\end{lemma}
\begin{proof}
	Considérons la réunion
	\begin{equation*}
		\bigcup_{\vec{v} \in \Gamma_G \cap B} (\vec{v} + \mathbf{T}).
	\end{equation*}
	Comme $\mathbf{T}$ est une tuile, pour tout $\vec{v} \in \Gamma_G \cap B$,
	$\vec{v} + \mathbf{T}$ est aussi une tuile.
	Mais comme dans le Lemme \ref{lem:tuiles-boule}, cette réunion est
	contenue dans la boule ${B(\vec{x}, r + 2R_0)}$, et donc le nombre de
	ces tuiles est fini et donc $\Gamma_G$ est fini.
\end{proof}

% lem:reseau-domaine-fondamental
\begin{lemma}
	\label{lem:reseau-domaine-fondamental}
	Soit $\Gamma = \Z\vec{u} + \Z\vec{v}$ un réseau, alors l'ensemble
	\begin{equation*}
		\mathcal{D}_\Gamma :=
		\{\alpha\vec{u} + \beta\vec{v}\ |\ \alpha, \beta \in [-1/2, 1/2[\}
	\end{equation*}
	est un domaine fondamental pour l'action de $\Gamma$ sur $\R^2$ par
	translation.
\end{lemma}

\begin{proof}
	Soit $\vec{x} \in \R^2$, alors $\vec{x}$ se décompose en une combinaison
	linéaire des vecteurs $\vec{u}$, et $\vec{v}$ (car ils forment une base).
	Donc il existe $\lambda_1, \lambda_2 \in \R$ tels que
	\begin{equation*}
		\vec{x} = \lambda_1 \vec{u} + \lambda_2 \vec{v}.
	\end{equation*}
	On pose $n_i = \lfloor \lambda_i + 1/2 \rfloor$ pour $i \in \{1, 2\}$, alors 
	\begin{equation*}
		\lambda_i - n_i = \lambda_i - \lfloor \lambda_i + 1/2 \rfloor
	\end{equation*}
	Mais puisque $x - 1 < \lfloor x \rfloor \leq x$ pour tout $x \in \R$, on
	obtient
	\begin{equation*}
		-1/2 \leq \lambda_i - n_i < 1/2	
	\end{equation*}
	Ainsi $\vec{x}_0 = (\lambda_1 - n_1)\vec{u} + (\lambda_2 - n_2)\vec{v}$
	$\in \mathcal{D}_g$ et $\vec{x} \in \Gamma + \vec{x}_0$.

	Soit $\vec{x}_0' \in \mathcal{D}_G$ tel que $\vec{x} \in \Gamma + \vec{x}_0'$,
	et soient $m_1, m_2 \in \Z$ tels que
	\begin{equation*}
		\vec{x} = \vec{x}_0' + m_1\vec{u} + m_2\vec{v}
	\end{equation*}
	Alors on a
	\begin{equation*}
		\vec{x}_0' + m_1\vec{u} + m_2\vec{v} = \vec{x}_0 + n_1\vec{u} + n_2\vec{v}
	\end{equation*}
	Soient $\alpha, \alpha', \beta, \beta' \in [-1/2, 1/2[$ tels que
	$\vec{x_0} = \alpha\vec{u} + \beta\vec{v}$ et
	$\vec{x}_0' = \alpha'\vec{u} + \beta'\vec{v}$, alors on obtient
	\begin{equation*}
		(\alpha' - \alpha + m_1 - n_1)\vec{u} + (\beta'-\beta + m_2 - n1)\vec{v} = 0
	\end{equation*}
	Puisque $\vec{u}$ et $\vec{v}$ sont libres, les coefficients doivent être
	égaux à 0. mais puisque $\alpha' - \alpha \in ]-1, 1[$ et $m_1 - n_1 \in \Z$,
	ceci implique que $\alpha' - \alpha = 0$. De même $\beta' - \beta = 0$.
	Donc $\vec{x}_0' = \vec{x}_0$, et ainsi pour tout $\vec{x} \in \R^2$,
	il existe un unique $\vec{x}_0 \in \mathcal{D}_\Gamma$ t.q.
	$\vec{x} \in \Gamma + \vec{x_0}$.
\end{proof}

% thm:translations-reseau
\begin{theorem}
	\label{thm:translations-reseau}
	Soit $\mathcal{P} = \mathcal{P}(\mathbf{P}, \Psi)$ un pavage régulier, et
	$G$ son groupe d'isometries, alors $\Gamma_G$ est un reseau.
\end{theorem}

\begin{proof}
	Comme le pavage est régulier, il existe $\vec{a}, \vec{b} \in \Gamma_G$ qui
	sont linéairement indépendants. Soit une boule ouverte $B = B(\vec{0}, r)$
	contenant $\vec{a}, \vec{b}$ alors par le Lemme \ref{lem:translations-boule},
	$B \cap \Gamma_G$ est fini.
	Il existe donc $\vec{u} \in \Gamma_G$ non-nul et de longueur minimale parmi
	les elements de $\Gamma_G$ et un element
	$\vec{v} \in \Gamma_G \setminus \R\vec{u}$ de longueur minimal parmi
	les elements de $\Gamma_G \setminus \R\vec{u}$.

	On pose $\Gamma = \Z\vec{u} + \Z\vec{v}$, alors clairment
	$\Gamma \subset \Gamma_G$. On va montrer que $\Gamma = \Gamma_G$.
	Posons
	\begin{equation*}
		\mathcal{D}_\Gamma = \{\alpha\vec{u} + \beta\vec{v}\ |\ 
		\alpha, \beta \in [-1/2, 1/2[\}.
	\end{equation*}
	Par le Lemme \ref{lem:reseau-domaine-fondamental}, c'est un domaine fondamental
	de $\Gamma$.

	Raisonnons par l'absurde et supposons qu'il existe $\vec{x} \in \Gamma_G$
	tel que $\vec{x} \notin \Gamma$. Puisque $\vec{u}$ et $\vec{v}$ sont
	linéairement indépendants, ils forment une base de $\R^2$ et donc, il existe 
	$\lambda_1, \lambda_2 \in \R$ tels que
	$\vec{x} = \lambda_1 \vec{u} + \lambda_2 \vec{v}$. Ainsi il existe
	${n_1, n_2 \in \Z}, {\mu_1, \mu_2 \in [-1/2, 1/2[}$ tels que
	\begin{equation*}
		\vec{x} = (\mu_1 + n_1)\vec{u} + (\mu_2 + n_2)\vec{v}
	\end{equation*}
	Puisque $\Gamma_G$ est un groupe additif,
	$\vec{y} = \mu_1\vec{u} + \mu_2\vec{v} \in \Gamma_G$, mais on a que
	\begin{equation*}
		\|\vec{y}\| \leq |\mu_1|\|\vec{u}\| + |\mu_2|\|\vec{v}\|
		\leq \frac{1}{2}\|\vec{u}\| + \frac{1}{2}\|\vec{v}\| \leq \|\vec{v}\|
	\end{equation*}
	On a alors 2 cas:
	\begin{enumerate}
		\item	$\|\vec{y}\| < \|\vec{v}\|$, mais par définition de $\vec{v}$, ceci
			implique que $\vec{y} \in \R\vec{u}$, et donc $\mu_2 = 0$, mais donc
			$\|\vec{y}\| \leq \frac{1}{2}\|\vec{u}\| < \|\vec{u}\|$, mais puisque
			$\vec{y} \notin \Gamma, \vec{y} \neq \vec{0}$, ce qui
			est en contradiction avec la minimalité de $\vec{u}$.
		\item $\|\vec{y}\| = \|\vec{v}\|$, mais puisque $\vec{u}$ et $\vec{v}$
			ne sont pas colinéaires, ceci implique que soit $|\mu_1|\|\vec{u}\| = 0$,
			soit $|\mu_2|\|\vec{v}\| = 0$. Si $|\mu_1|\|\vec{u}\| = 0$,
			alors $\vec{y} \notin \R\vec{u}$, car $\vec{y} \neq 0$, et donc
			$\|\vec{y}\| = \frac{1}{2}\|\vec{v}\|$ ce qui contredit la minimalité
			de $\vec{v}$. Sinon $|\mu_2|\|\vec{v}\| = 0$ et donc
			$\|\vec{y}\| = \frac{1}{2}\|\vec{u}\|$ ce qui contredit la minimalité
			de $\vec{u}$.
	\end{enumerate}
	Ainsi $\Gamma_G \subset \Gamma$ et donc $\Gamma_G = \Gamma$.
\end{proof}


% sssection:groupe-rotations-lineaires
\subsubsection{Le groupe des rotations linéaires}

\begin{notation}
	Soit $G$ le groupe d'isometrie d'un pavage regulier $\mathcal{P}$. On notera
	$G_0 = \lin(G)$ et $G_0^+ = \lin(G^+)$ les images de $G$ et $G^+$
	respectivement par le morphisme de groupes
	$\lin: \Isom(\R^2) \to \Isom(\R^2)_0$.
	Dans cette partie on identifiera $G_0^+$ avec un sous-groupe de $\C^1$.
	Finalement, on notera $\Gamma$ pour $\Gamma_G$
\end{notation}

% lem:invariance-rotation
\begin{lemma}
	\label{lem:invariance-rotation}
	Soit $\mathcal{P}$ un pavage régulier et $G$ son groupe d'isometries, pour
	tout $r \in G^+$ d'angle $\alpha \in \C^1$, $\alpha\Gamma = \Gamma$.
\end{lemma}

\begin{proof}
	Si $G^+ = T_\Gamma$, alors $\alpha = 1$ et c'est clair,
	sinon $G^+ \neq T_\Gamma$, et donc $G_0^+ \neq {1}$.
	Soit $r_{\alpha, \nu} \in G^+$, soit $t_\gamma \in T_\Gamma$, alors
	$\Ad(r)(t_\gamma) = r \circ t_\gamma \circ r^{-1} = t_{\alpha\gamma}$,
	donc $\alpha\gamma \in \Gamma$ et donc $\alpha\Gamma \subset \Gamma$.
	En particulier $\alpha^{-1} \in G_0^+$, donc $\alpha^{-1} \subset \Gamma$
	et ainsi $\alpha \Gamma = \Gamma$.
\end{proof}

% thm:pavage-rotations
\begin{theorem}
	\label{thn:pavage-rotations}
	Soit $\mathcal{P}$ un pavage régulier et $G$ son groupe d'isometries,
	alors 
	\begin{equation*}
		G_0^+ \subset \{\mu_1, \mu_2, \mu_3, \mu_4, \mu_6\}
	\end{equation*}
	où $\mu_n = \{z \in \C^1\ |\ z^n = 1\}$ est l'ensemble des racines n-ièmes de
	l'unité. De plus, il existe $\gamma_0 \in \C^*$ tel que
	\begin{itemize}
		\item Si $G_0^+ \in \{\mu_3, \mu_6\}$: $\Gamma = \gamma_0(\Z + \Z\omega_3)$
			où $\omega_3 = -\frac{1}{2} \pm i \frac{\sqrt{3}}{2}$
		\item Si $G_0^+ = \mu_ 4$: $\Gamma = \gamma_0(\Z + \Z i)$
	\end{itemize}
\end{theorem}

\begin{proof}
	On a que $\Gamma = \Z\gamma_0 + \Z\gamma_1$ pour $\gamma_0, \gamma_1 \in \C^*$, 
	comme dans la preuve du Theorème \ref{thm:translations-reseau}
	avec $|\gamma_1| \geq |\gamma_0|$, et on pose
	$\Gamma ' = \frac{1}{\gamma_0} \Gamma = \Z + \Z\gamma$, où
	$\gamma = \frac{\gamma_1}{\gamma_0}$. $\Gamma'$ est un reseau obtenu
	à partir de $\Gamma$ par la multiplication par
	$\gamma_0^{-1} = |\gamma_0|^{-1}\frac{|\gamma_0|}{\gamma_0}$,
	ce qui est une composition de rotation linéaire d'angle
	$\frac{|\gamma_0|}{\gamma_0} \in \C^1$ et l'homothetie de rapport
	$|\gamma_0|^{-1}$. Par notre choix de $\gamma_0$ et $\gamma_1$, l'élement de module
	minimal les elements de $\Gamma' \setminus \{0\}$ c'est 1, et
	$\gamma$ est un element de module minimal dans $\Gamma \setminus \R$.
	On a de plus $|\gamma| \geq 1$

	On a alors (en invoquant le Lemme \ref{lem:invariance-rotation})
	pour tout $\alpha \in G_0^+$
	\begin{equation*}
		\alpha\Gamma' = \frac{\alpha}{\gamma_0}\Gamma =
		\frac{1}{\gamma_0}(\alpha\Gamma) = \frac{1}{\gamma_0}\Gamma = \Gamma'
	\end{equation*}
	En particulier, comme $1 \in \Gamma'$, $\alpha \in \Gamma'$ et donc
	$G_0^+ \in \Gamma'$. Comme $\alpha^{-1} \in G_0^+$ et que
	$\alpha^{-1} = \overline{\alpha}$, on a que 
	\begin{equation*}
		\alpha + \alpha^{-1} = \alpha + \overline{\alpha} =
		2 \Re(\alpha) \in \R \cap \Gamma' = \Z.
	\end{equation*}
	Comme $|2\Re(\alpha)| \leq 2$, on a $2\Re(\alpha) \in \{-2, -1, 0, 1, 2\}$.
	
	On a donc les cas suivants:
	\begin{itemize}
		\item Si $2\Re(\alpha) = \pm 2$, alors $\alpha = \pm 1$.
		\item Si $2\Re(\alpha) = \pm 1$, alors $\alpha = \pm\frac{1}{2} \pm
			i\frac{\sqrt{3}}{2}$, et donc $\alpha$ est bien une
			racine primitive 3-ème de
			l'unité $(\alpha = -\frac{1}{2} \pm i\frac{\sqrt{3}}{2})$, ou
			une racine primitive 6-ème de l'unité
			$(\alpha = \frac{1}{2} \pm i\frac{\sqrt{3}}{2})$
		\item Si $2\Re(\alpha) = 0$, alors $\alpha = \pm i$, et donc $\alpha$ est
			bien une racine primitive 4-ème de l'unité.
	\end{itemize}
	En particulier $G_0^+$ est un groupe fini, donc cyclique et
	si $\omega$ est un générateur, son ordre est 1, 2, 3, 4 ou 6,
	donc $\omega \in \{1, -1, \omega_3^{\pm 1}, i^{\pm 1}, \omega_6^{\pm 1}\}$
	et puisque $G_0^+ \subset \Gamma'$, on a que
	$G_0^+ \in \{\mu_1, \mu_2, \mu_3, \mu_4, \mu_6\}$.
	On note aussi que si $\mu_3 \in \Gamma'$, alors $\mu_6 \in \Gamma'$,
	car $\omega_3 \in \Gamma' \implies -\omega_3 \in \Gamma'$, qui est d'ordre 6.

	Supposons que $G_0^+$ soit d'ordre 3, 4 ou 6, alors $\Gamma'$
	contient suivant les cas où bien $\Z + \Z i$, où bien $\Z + \Z\omega_3$.
	Montrons que dans ces cas, ces inclusions sont des égalités.
	On a que $\Gamma = \Z + \Z\gamma$ où $\gamma$ est de module minimal parmi les
	élements de $\Gamma \setminus \R$. On a donc $|\gamma| \geq 1$, mais comme
	$|i| = |\omega_3| = 1$, $i$ et $\omega_3$ sont suivant le cas de module
	minimal parmi les elements de $\Gamma \setminus \R$.
	On en déduit (suivant le cas)
	\begin{equation*}
		\Gamma' = \Z + \Z i \textrm{ ou bien } \Gamma' = \Z + \Z\omega_3
	\end{equation*}



\end{proof}

% thm:fedorov-rotations
\begin{theorem}[Theorème de Fedorov pour les groupes de rotations]
	Il existe 5 classe d'isomorphismes possibles poue le sous-groupe des
	isometries directes $G^+ \subset G$ d'un groue cristallographique.
\end{theorem}
\begin{proof}
	Soit $r$ une rotation de $G^+$ d'angle $\omega$ d'ordre maximal
	et de centre $z_r \in \C$.
	Appliquant au pavage $\mathcal{P} = \mathcal{P}(\mathbf{T}, \Phi)$ la
	translation $t_{-z_r}$, on obtient le pavage
	$\mathcal{P}' = \mathcal{P}(\mathbf{T}', \Phi')$, avec
	$\mathbf{T}' = \mathbf{T} - z_r$ dont le groupe d'isometries est le groupe
	conjugué
	\begin{equation*}
		G' = \Ad(t_{-z_r})(G) \textrm{ et } G'^+ = \Ad(t_{-z_r})(G^+) 
	\end{equation*}
	et on a 
	\begin{equation*}
		T_{G'} = \Ad(t_{-z_r})(T(\Gamma)) = T(\Gamma)
	\end{equation*}
	D'autre part
	\begin{equation*}
		r' = \Ad(t_{-z_r})(r) = r_{\omega, 0}
	\end{equation*}
	car $z_r$ est le centre de $r$. Ainsi quitte a remplacer le pavage
	$\mathcal{P}$ par le pavage $\mathcal{P}'$, (de même pour les groupes des
	isometries) ainsi $r_{\omega, 0} \in G^+$. et
	$G^+ = \langle r_{\omega, 0} \rangle$.

	Écrivant $\Gamma = \Z\gamma_0 +\Z\gamma_1$, comme dans la preuve
	\ref{thm:translations-reseau};
	la transformation linéaire
	\begin{equation}
		[\times \frac{1}{\gamma_0}]: z \mapsto \frac{1}{\gamma_0} z
	\end{equation}
	est la composée d'une rotation linéaire (d'angle
	$\frac{|\gamma_0|}{\gamma_0}$) et d'une homothetie linéaire (de rapport 
	$\frac{1}{|\gamma_0|}$) et transforme le pavage
	$\mathcal{P}(\mathbf{T}, \Phi)$ en
	\begin{equation*}
		\mathcal{P}' = \mathcal{P}(\mathbf{T}', \Phi') =
		\left(\frac{1}{\gamma_0}\mathbf{T},
		\Ad\left(\frac{1}{\gamma_0}\right)\Phi\right)
	\end{equation*}
	Les sous-groupes associés au groupe d'isometries $G'$ sont les conjugués par
	$\frac{1}{\gamma_0}$. Ainsi quitte à remplacer $G$ par $G'$ et on peut
	supposer que $G$ vérifie $G_0^+ \subset G^+$ et $\Gamma=\Z + \gamma\Z$ avec
	$\gamma$ de module $|\gamma| \geq 1$ et de longueur minimal parmi les
	elements de $\Gamma \setminus \R$. Comme $G_0^+ \subset G^+$, tout élement
	$r$ de $G^+$ s'écrit de manière unique sous la forme
	\begin{equation*}
		r = t_z \circ r_{\alpha, 0}, z\in\C, \alpha \in G_0^+,
	\end{equation*}
	comme $G_0^+ \subset G^+$ il en résulte que
	$t_z = r \circ r_{\alpha, 0}^{-1}$ et donc $z \in \Gamma$.
	Ainsi $G^+$ est engendré par $T(\Gamma)$ et par $G_0^+$ et est donc, soit le
	groupe engendré par
	\begin{enumerate}
		\item le réseau de translations $T(\Gamma)$ pour $\Gamma \in \C$ un
			réseau,
		\item la rotation linéaire $r_{-1, 0}$ et le
			réseau des translations $T(\Gamma)$,
		\item la rotation $r_{\omega_3, 0}$ et le réseau de translations
			$\Gamma = \Z + \Z\omega_3$
		\item la rotation $r_{i, 0}$ et le réseau de translations
			$\Gamma = \Z + \Z i$
		\item la rotation $r_{\omega_6, 0}$ et le réseau de translations
			$\Gamma = \Z + \Z\omega_3$
	\end{enumerate}
\end{proof}

\begin{definition}
	Soit $G$ un groupe cristallographique contenu dans $\Isom(\R^2)^+$
	(ie. $G = G^+$) et $\mathcal{P}$ un pavage associé.
	On dit que $\mathcal{P}$ est de type $p1, p2, p3, p4$ ou $p6$, suivant
	l'ordre maximal des rotations dans $G^+$.
\end{definition}


% thm:fedorov
\begin{theorem}[Theorème de Fedorov]
	Il existe 17 classes d'isomorphismes possiles pour un groupe
	cristallographique.
\end{theorem}

\subsection{Groupes paveurs}

\begin{definition}
	Soit $G \in Isom(R^2)$ un \emph{sous-groupe} d'isometries tel qu'il existe une tuile
	$\mathcal{P}$ telle que
	\begin{equation*}
		\mathcal{P}(\mathbf{T}, \Phi) = \{g \cdot \mathbf{T}, g \in G\}
	\end{equation*}
	forme un pavage du plan. On dit alors que $G$ est une groupe paveur.
\end{definition}

\begin{remark}
	Si $\mathcal{P}$ est un pavage associé à un groupe paveur $G$, alors $G$ est
	un sous-groupe du groupe des isometries de ce pavage.
\end{remark}

\begin{remark}
	$G$ peut être inclus strictement dans $G_\mathcal{P}$
\end{remark}

\begin{theorem}
	Soit $G$ un groupe paveur et $T_G$ son sous-groupe des translations.
	Alors $T_G = T(\Gamma)$ est un réseau de translations: il existe
	$\gamma_0, \gamma_1 \in \Gamma$, $\R$-linéairement indépendants tels que
	$\Gamma = \Z\gamma_0 + \Z\gamma_1$. En particulier $\mathcal{P}$ est un pavage
	régulier.
\end{theorem}

\begin{proof}
	Montrons que l'ensembledes vecteurs de translations $\Gamma$ contient deux
	élements $\R$-linéairement indépendants: on en déduit que $\Gamma$ est un
	réseau en suivant la preuve du Théorème \ref{thm:translations-reseau} et
	comme $T(\Gamma) \subset G \subset G_\mathcal{P}$, $\mathcal{P}$ est un
	pavage régulier, $G$ possède un nombre infini d'élements, sinon $\mathcal{P}$ ne peut pas
	couvrir $\R^2$. On en déduit que $G^+$ est infini, car
	$G = G^+ \sqcup sG^+$ où $s \in G \setminus G^+$.

	Supposons que $\Gamma = \{0\}$ et que tous les élements de $G^+$ soient des
	rotations de même centre, alors $G \cdot \mathbf{T}$ est un ensemble de copies de
	$\mathbf{T}$ tournant autour de ce centre et ainsi est entièrement contenu
	dans une boule. Donc dans ce cas $G \cdot \mathbf{T}$
	ne peut pas recouvrir $\R^2$.

	Supposons que $\Gamma = \{0\}$ et que $G^+$ contient deux rotations $r$ et
	$r'$ non triviales et de centres distincts $z \neq z'$.
	Soient $\alpha, \alpha'$ leurs angles, alors le commutateur
	\begin{equation*}
		[r, r'] = rr'r^{-1}r'^{-1}
	\end{equation*}
	est une rotation d'angle
	$[\alpha, \alpha'] = \alpha \alpha' \alpha^{-1} \alpha'^{-1} = 1$ et est donc
	une translation. Le vecteur de translation est obtenu en calculant l'image
	d'un point, par exemple $z'' = r'(z)$:
	\begin{equation*}
		[r,r'](r'(z)) = rr'r^{-1}(z) = r(r'(z))
	\end{equation*}
	et donc
	\begin{equation*}
		[r, r'] = t_{r(r'(z)) - r'(z)}
	\end{equation*}
	Si le vecteur $\gamma = r(r'(z)) - r'(z)$ était nul, $r'(z)$ serait un point
	fixe de $r$ donc égal à son centre $z$ et $z$ serait un point fixe de $r'$
	et donc égal à $z'$. Ainsi $\gamma \neq 0$ et $t_\gamma \in T_G$.

	On a donc montré que $G^+$ contient au moins une translation non-triviale
	$t_\gamma$.

	Supposons qu'il n'y en ait pas d'autre et que $G^+ = T_G$ soit contenu dans
	le groupe des translations des vecteurs colineaires à $\gamma$,
	$T(\R\gamma)$. Si $G = G^+$, c'est impossible car alors $G \cdot \mathbf{T}$
	serait contenu dans une bande délimitée par deux droites paralelles à
	$\gamma$.

	Si $G \neq G^+$, alors
	\begin{equation*}
		G \subset T(\R\gamma) \sqcup sT(\R\gamma)
	\end{equation*}
	et $G \cdot \mathbf{T}$ serait contenu dans deux bandes
	$T(\R\gamma) \cdot \mathbf{T}$ et $s(T(\R\gamma) \cdot \mathbf{T})$ ce qui à
	nouveau ne couvre pas $\R^2$.

	On a donc montré que $G^+$ n'est pas contenu dans $T(\R \gamma)$.

	Si $G^+$ contient une translation $t_\gamma'$ avec $\gamma' \notin \R\gamma$
	on a fini, car $\Gamma$ contient deux élements $R$-linéairement
	indépendants. Sinon $G^+$ contient une translation affine $r_{\alpha, \nu}$
	qui n'est pas une translation (i.e son angle $\alpha \neq 1$). On a alors
	\begin{equation*}
		r_\alpha \circ t_\gamma \circ r_\alpha^{-1} = t_{r_\alpha(\gamma)} \in G^+
	\end{equation*}
	En effet (passant aux complexes)
	\begin{equation*}
		r_\alpha \circ t_\gamma \circ r\alpha^{-1} = \alpha(\gamma + \alpha^{-1}z)
		= \alpha\gamma + z = t_{\alpha\gamma}(z)
	\end{equation*}
	donc si $\alpha \neq -1, \gamma' = r_\alpha(\gamma)$ n'est pas colinéaire à
	$\gamma$ et $T_G$ contient $t_\gamma$ et $t_{\gamma'}$.
	Finalement, si les seules rotations non-triviales sont d'angles $\alpha =
	-1$, on retrouve que $G\cdot\mathbf{T}$ est contenu dans la réunion de
	quatre bandes ce qui ne permet pas de couvrir le plan $\R^2$
\end{proof}

\begin{lemma}
	\label{lem:sous-groupes-finis}
	Soit $G$ un groupe et $K \subset H \subset G$ des sous-groupes. Alors le
	conditions suivantes sont equivalentes:
	\begin{enumerate}
		\item $G/K$ est fini.
		\item $G/H$ et $H/K$ sont finis.
	\end{enumerate}
	De plus, on a la relation entre cardinaux
	\begin{equation*}
		|G/K|=|G/H||H/K|
	\end{equation*}
\end{lemma}

\begin{proposition}
	Soit $\Gamma' \subset \Gamma \subset \R^2$ deux réseaux contenus l'un dans
	l'autre alors le groupe quotient ($\Gamma$ est commutatif) $\Gamma/\Gamma'$
	est fini.
\end{proposition}

\begin{proof}
	Comme $\Gamma \supset \Gamma'$ il existe $a, b, c, d \in Z$ t.q.
	\begin{equation*}
		\gamma_0' = a \gamma_0 + b \gamma_1\textrm{ et }
		\gamma_1' = c \gamma_0 + d \gamma_1
	\end{equation*}
	mais comme $(\gamma_0', \gamma_1')$ est une base de $\R^2$, la matrice
	\begin{equation*}
		\begin{bmatrix}
			a & b \\
			c & d
		\end{bmatrix}
	\end{equation*}
	est inversible d'inverse
	\begin{equation*}
		\begin{bmatrix}
			a' & b' \\
			c' & d'
		\end{bmatrix}
		= \frac{1}{ad - bc}
		\begin{bmatrix}
			d & -b \\
			-c & a
		\end{bmatrix}
	\end{equation*}
	et $a', b', c', d' \in \Q$. Posant $\Delta = ad - bc$, on a
	\begin{equation*}
		\gamma_0 = \frac{d}{\Delta}\gamma_0' + \frac{-b}{\Delta}\gamma_1'\textrm{ et }
		\gamma_1 = \frac{-c}{\Delta}\gamma_0' + \frac{a}{\Delta}\gamma_1'
	\end{equation*}
	Ainsi
	\begin{equation*}
		\Gamma' \subset \Gamma \subset \Gamma'' =
		\Z \frac{\gamma_0'}{\Delta} + \Z \frac{\gamma_1'}{\Delta}
	\end{equation*}
	On peut alors écrire $\Gamma' = \Delta\Gamma''$ et en posant
	\begin{equation*}
		\gamma_0'' = \Delta^{-1} \gamma_0' \textrm{ et }
		\gamma_1'' = \Delta^{-1} \gamma_1'
	\end{equation*}
	on a
	\begin{equation*}
		\Gamma' = \Delta\Z\gamma_0'' + \Delta\Z\gamma_1'' \subset
		\Z\gamma_0'' + \Z\gamma_1''
	\end{equation*}
	L'application
	\begin{equation*}
		(m, n) \in \Z^2 \mapsto m \gamma_0'' + n\gamma_1'' \in \Gamma''
	\end{equation*}
	est un isomorphisme de groupe qui identifie $\Delta\Z^2$ à $\Gamma'$. On a
	donc que 
	\begin{equation*}
		\Gamma''/\Gamma' \cong \Z^2/\Delta\Z^2 = (\Z/\Delta\Z)^2
	\end{equation*}
	qui est donc fini d'ordre $|\Delta|^2$. Il s'ensuit par le Lemme
	\ref{lem:sous-groupes-finis} que $\Gamma/\Gamma'$ est fini.
\end{proof}
