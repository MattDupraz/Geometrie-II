\documentclass[french]{article}
\usepackage[utf8]{inputenc}
\usepackage{babel}
\usepackage[T1]{fontenc}
\usepackage{amsthm}
\usepackage{amsmath}
\usepackage{amsfonts}
\usepackage{amssymb}

\usepackage{hyperref}
\hypersetup{
	colorlinks,
	allcolors=black
}

\theoremstyle{plain}
\newtheorem{theorem}{Theorème}[section]
\newtheorem{corollary}{Corollaire}[theorem]
\newtheorem{lemma}[theorem]{Lemme}
\newtheorem{proposition}[theorem]{Proposition}

\theoremstyle{remark}
\newtheorem*{remark}{Remarque}

\theoremstyle{definition}
\newtheorem{definition}{Définition}[section]
\newtheorem*{notation}{Notation}

% Notations
\newcommand{\R}{\mathbb{R}}
\newcommand{\C}{\mathbb{C}}

\DeclareMathOperator{\Bij}{Bij}
\DeclareMathOperator{\End}{End}
\DeclareMathOperator{\Aut}{Aut}
\DeclareMathOperator{\GL}{GL}
\DeclareMathOperator{\Hom}{Hom}
\DeclareMathOperator{\Ad}{Ad}
\DeclareMathOperator{\id}{id}
\DeclareMathOperator{\red}{red}
\DeclareMathOperator{\I}{I}
\DeclareMathOperator{\im}{im}

\newcommand{\actionl}{\curvearrowright}
\newcommand{\actionr}{\curvearrowleft}
\newcommand{\field}[1]{\bathbb{#1}}

\begin{document}

\title{Géometrie II \\
	\large Resumé basé sur le polycopié de Prof. Philippe Michel}
\author{Matthew Dupraz}

\maketitle

\newpage

\tableofcontents

\newpage

\section{Action d'un groupe sur un ensemble}
\subsection{Action à gauche, à droite}

% Action à gauche
\begin{defn}
	Une \emph{action à gauche} d'un groupe $G$ sur un ensemble $E$ est une application
	\begin{align*}
		G \times E &\to E \\
		(g, x) &\mapsto g \cdot x,
	\end{align*}
	vérifiant les propriétés suivantes:
	\begin{align*}
		& \forall x \in E: e_G \cdot x = x, \\
		& \forall g, g' \in G, \forall x \in E:
			g' \cdot (g \cdot x) = (g' g) \cdot x.
	\end{align*}
\end{defn}

% Action à droite
\begin{defn}
	Une \emph{action à droite} d'un groupe $G$ sur un ensemble $E$ est une application
	\begin{align*}
		G \times E &\to E \\
		(g, x) &\mapsto x \cdot g,
	\end{align*}
	vérifiant les propriétés suivantes:
	\begin{align*}
		& \forall x \in E: x \cdot e_G = x, \\
		& \forall g, g' \in G, \forall x \in E:
			(x \cdot g) \cdot g' = x \cdot (g g').
	\end{align*}
\end{defn}

% Morphisme associé
\begin{defn}
	Le morphisme de groupes $\phi : G \to \Bij(E)$ \emph{associé} à l'action d'un
	groupe $G$ sur un ensemble $E$ est un morphisme tel que $g \cdot x = (\phi(g))(x)$
	(resp. $x \cdot g = (\phi(g))(x)$ pour les actions à droite)
	pour tout $g \in G, x \in E$.
\end{defn}

% Notation "agit"
\begin{notn}
	On note $G \agitg E$ le fait qu'un groupe agisse sur un ensemble $E$.
	Pour spécifier qu'il s'agit d'une action à droite on note $E \agitd G$.
\end{notn}

% G-ensembles
\begin{notn}
	Soit $G$ un groupe et $E$ un ensemble, tel que $G \agitg E$.
	On appele alors $E$ un \emph{$G$-ensemble}
\end{notn}

% Noyau
\begin{defn}
	Soit $G$ un groupe agissant sur un ensemble $E$,
	alors le \emph{noyau} de l'action est le noyau $\ker(\phi)$ du morphisme
	$\phi : G \to \Bij(E)$ associé. De même, l'\emph{image}
	de l'action est l'image $\im(\phi)$ du morphisme associé.
\end{defn}

% Morphisme de G-ensembles
\begin{defn}
	Soit $G$ un groupe. Un \emph{morphisme de $G$-ensembles} est une
	application $\psi : X \to Y$ qui est compatible avec
	la structure des $G$-ensembles $X$ et $Y$ --- elle satisfait
	\begin{equation*}
		\forall g \in G, \forall x \in X:
			\psi(g \cdot x) = g \times \psi(x).
	\end{equation*}
	On appelle $\psi$ aussi un \emph{$G$-morphisme},
	ou une \emph{application d'entrelacement}.
\end{defn}

\subsection{Orbites, stabilisateurs et points fixes}

% Orbite
\begin{defn}
	L'\emph{orbite} d'un élément $x \in E$ sous l'action de $G$
	(ou encore une \emph{$G$-orbite}) est l'ensemble
	\begin{equation*}
		G \cdot x := \{g \cdot x\ |\ g \in G\}.
	\end{equation*}
\end{defn}

\begin{notn}
	On note $x \cdot G$ l'orbite de $x$ si $E \agitd G$.
\end{notn}

\begin{defn}
	Soit $G$ un groupe et $E$ un $G$-ensemble. 
	On note
	\begin{equation*}
		E / G := \{ G \cdot x\ |\ x \in E\}, 	
	\end{equation*}
	l'ensemble des $G$-orbites de $E$. On appelle cet
	ensemble le \emph{quotient de $E$ sous l'action de $G$}.
\end{defn}

\begin{notn}
	On peut noter $G \backslash E$ pour spécifier qu'il s'agit d'une action à gauche.
\end{notn}

% Points fixes
\begin{defn}
	L'ensemble des \emph{points fixes} d'un element $g \in G$ est l'ensemble
	\begin{equation*}
		E^g := \{x \in E\ |\ g \cdot x = x\}.
	\end{equation*}
\end{defn}

% Stabilisateur
\begin{defn}
	Le \emph{stabilisateur} d'un élément $x \in E$ sous l'action de $G$ est l'ensemble
	\begin{equation*}
		G_x := \{g \in G\ |\ g \cdot x = x\}.
	\end{equation*}
\end{defn}

\begin{rmq}
	Le stabilisateur d'un element $x \in E$ sous l'action de $G$ est un sous-groupe de $G$.	
\end{rmq}

% Orbites - relation d'équivalence
\begin{thm}
	Soit $E$ un ensemble et $G$ un groupe agissant sur $E$. La relation définie par
	\begin{equation*}
		x \sim y \iff x \in G \cdot y,
	\end{equation*}
	pour $x, y \in E$, est une relation d'équivalence.
\end{thm}

\begin{proof}
	Cette relation est:
	\begin{itemize}
		\item Réflexive, car pour $x \in E$, on a $x = e_G \cdot x$ et donc $x \sim x$.
		\item Symétrique, car pour $x, y \in E$, si $x \sim y$,
			alors il existe $g \in G$ tel que $x = g \cdot y$, et donc $y = g^{-1} \cdot x$,
			d'où $y \in G \cdot x$ et $y \sim x$.
		\item Transitive, car pour $x, y, z \in E$ si $x \sim y$ et $y \sim z$,
			alors il existe $g,g' \in G$ tel que $x = g \cdot y$ et
			$y = g' \cdot z$ et donc $x = (g g') \cdot z$ et $x \sim z$.
	\end{itemize}
\end{proof}

\begin{rmq}
	Les classes d'equivalence de $\sim$ sont les orbites $G \cdot x$, $x \in E$.
\end{rmq}

\begin{cor}
	Le quotient d'un $G$-ensemble $E$ sous l'action de $G$ (l'ensemble de $G$-orbites de $E$)
	forme une partition de $E$.
\end{cor}

\begin{cor}[Formule des classes]
	Si $E$ est un enseble fini, alors
	\begin{equation*}
		|E| = \sum_{A \in E/G}|A|
	\end{equation*}
\end{cor}

\begin{defn}
	On note $\pi_G$ ou $\red_G$ l'application surjective
	\begin{align*}
		red_G : E &\to E/G \\
		x &\mapsto G \cdot x
	\end{align*}
	On appelle cette application \emph{projection sur $E/G$} ou encore
	\emph{reduction modulo $G$}.
\end{defn}

\begin{defn}
	Soit $G$ un groupe et $E$ un $G$-ensemble.
	Un \emph{domaine fondamental} est un sous-ensemble
	$\mathcal{D}_G \subset E$ tel que
	\begin{equation*}
		E = \bigsqcup_{x \in \mathcal{D}_G} G \cdot x
			= \bigsqcup_{g \in G} g \cdot \mathcal{D}_G
	\end{equation*}
\end{defn}

\begin{rmq}
	Un domaine fondamental n'est pas uniquement défini.
\end{rmq}

% theoreme-lagrange

\begin{thm}[Theorème de Lagrange]
	Soit $G$ un groupe et $H$ un sous-groupe de $G$
	qui agit sur $G$ par translation. Alors si $H$ est fini,
	\begin{equation*}
		\forall A \in G/H: |A| = |H|
	\end{equation*}
	Si de plus $G$ est fini, alors l'ordre de $H$ divise l'ordre de $G$,
	et 
	\begin{equation*}
		|G/H| = \frac{|G|}{|H|}
	\end{equation*}
\end{thm}

\begin{notn}
	Le cardinal $|G/H|$ s'appelle \emph{l'indice de H dans G} et se note
	$[G : H]$
\end{notn}

\begin{proof}
	Supposons $H$ fini, comme l'application de translation à droite par $g$ est injective,
	et que l'image de $H$ par cette application est $H\,g$,
	on a $|H\,g| = |H|$, et donc toutes les orbites sont de la même taille.\\
	Supposons que $G$ soit fini, par la formule des classes on a alors
	\begin{equation*}
		|G| = \sum_{A \in G/H}|A| = \sum_{A \in G/H}|H| = |G/H||H|
	\end{equation*}
	d'où il s'ensuit que
	\begin{equation*}
		[G:H] = |G/H| = \frac{|G|}{|H|}
	\end{equation*}
\end{proof}

% theoreme-orbite-stabilisateur

\begin{thm}[Theorème orbite-stabilisateur]
	Soit $G$ un groupe et $E$ un $G$-ensemble.
	Soit $x \in E$ et $G \cdot x$ son orbite.
	L'application
	\begin{align*}
		\psi : G/G_x &\xrightarrow{\sim} G \cdot x\\
		g\,G_x &\mapsto g \cdot x
	\end{align*}
	est un isomorphisme de $G$-ensembles.
	
\end{thm}

\begin{proof}
	L'application est bien définie, supposons qu'il existe $g, g' \in G$,
	tel que $g G_x = g' G_x$, alors il existe $h \in G_x$,
	tel que $g = g' h$. Il s'ensuit que
	\begin{equation*}
		\psi(g\,G_x) = g \cdot x = (g' h) \cdot x
		= g' \cdot (h \cdot x) = g' \cdot x = \psi(g'\,G_x)
	\end{equation*}
	Montrons que $\psi$ est bien bijective:
	\begin{itemize}
		\item Surjectivité: Soit $y \in A$, alors il existe $g \in G$ tel que $y = g \cdot x$,
			et puisque $g \cdot x = \psi(g\,G_x)$, on a bien que $y = \psi(g\,G_x)$
		\item Injectivité: Soient $g, g' \in G$ tels que
			$g \cdot x = g' \cdot x$. Alors on a que $x = (g^{-1} g') \cdot x$ et donc
			$g^{-1} g' \in G_x$ puis $g' \in g\,G_x$.
			En d'autres termes $g'\,G_x = g\,G_x$
	\end{itemize}
	Finalement montrons que $\psi$ est un isomorphisme de $G$-ensembles.
	Soient $g, g'\in G$, alors
	\begin{equation*}
		\psi(g'(g\,G_x)) = \psi((g' g) G_x)
			= (g' g) \cdot x = g' \cdot (g \cdot x) = g' \cdot \psi(g\,G_x)
	\end{equation*}
\end{proof}

\begin{cor}
	\label{formule-orbite-stabilisateur}
	Soit $G$ un groupe, $E$ un $G$-ensemble et $x \in E$. Alors si $E$ et $G$ sont finis, on a
	\begin{equation*}
		|G \cdot x| = |G/G_x| = \frac{|G|}{|G_x|}
	\end{equation*}
\end{cor}

\begin{proof}
	Par le theorème orbite-stabilisateur, $|G \cdot x|$ et $|G/G_x|$
	sont des $G$-ensembles isomorphes. Puisqu'ils sont finis,
	il s'ensuit que $|G \cdot x| = |G/G_x|$.
	Par le theorème de Lagrange, $|G/G_x| = |G|/|G_x|$.
\end{proof}

\begin{prop}
	\label{stab-conjugue}
	Soit $G$ un groupe et $E$ un $G$-ensemble, soit $x \in E$ et
	$G \cdot x$ l'orbite de $x$, alors pour tout $y \in G \cdot x$,
	$G_x$ est conjugué (donc isomorphe) à $G_y$.
\end{prop}

\begin{proof}
	Soit $y \in G \cdot x$, alors il existe $g \in G$ tel que $y = g \cdot x$.
	Soit $h \in G_y$, alors on a
	\begin{equation*}
		h \cdot y = y \iff h \cdot (g \cdot x) = g \cdot x \iff
		g^{-1} h g = x \iff g^{-1} h g \in G_x,
	\end{equation*}
	et ainsi $h \in \Ad(g)(G_x)$ et donc $G_y \subset \Ad(g)(G_x)$.
	Puisque $x \in G \cdot y$ et $x = g^{-1} \cdot y$ on a 
	aussi que $G_x \subset \Ad(g^{-1})(G_y)$ et donc $G_y = \Ad(g)(G_x)$
\end{proof}

% formule-burnside

\begin{thm}[Formule de Burnside]
	Soit $G$ un groupe fini et $E$ un $G$-ensemble fini. Alors
	\begin{equation*}
		|E/G| = \frac{1}{|G|}\sum_{g \in G}|E^g|.
	\end{equation*}
\end{thm}

\begin{proof}
	Posons $S = \{(g, x) \in G \times E\ |\ g \cdot x = x\}$, alors
	\begin{equation*}
		\sum_{g \in G} |E^g| = |S| = 
		\sum_{x \in X} |G_x| = \sum_{A \in E/G} \sum_{x \in A} |G_x|.
	\end{equation*}
	Soit $A \in E/G$, soit $x \in A$, alors par le Corollaire
	\ref{formule-orbite-stabilisateur}, $|G_x| = \frac{|G|}{|A|}$ et donc
	\begin{equation*}
		\sum_{A \in E/G} \sum_{x \in A} |G_x| = \sum_{A \in E/G} \sum_{x \in A}\frac{|G|}{|A|}
		\sum_{A \in E/G} |A|\frac{|G|}{|A|} = |E/G||G|.
	\end{equation*}
	D'où on obtient que
	\begin{equation*}
		|E/G| = \frac{1}{|G|}\sum_{g \in G}|E^g|.
	\end{equation*}
\end{proof}

\subsection{Caractéristiques des actions de groupe}

% Action fidele
\begin{defn}
	Une action $G \agitg E$ est dite \emph{fidèle} si pour $g \in G$,
	\begin{equation*}
		\phi(g) = \id_E \iff g = e_G,
	\end{equation*}
	où $\phi : G \to \Bij(E)$ est le morphisme associé à l'action.
\end{defn}

% Action libre
\begin{defn}
	Une action $G \agitg E$ est dite \emph{libre} si
	\begin{equation*}
		\forall x \in E, G_x = \{e_G\}.
	\end{equation*}
\end{defn}

\begin{rmq}
	Une action libre est fidèle.	
\end{rmq}

% Action transitive
\begin{defn}
	Une action $G \agitg E$ est dite \emph{transitive} si elle
	possède une et une seulle orbite, c'est à dire $G \cdot x = E$ pour tout $x \in E$.
\end{defn}

% Action n-transitive
\begin{defn}
	Une action $G \agitg E$ est dite \emph{n-transitive} si l'action induite sur
	l'ensemble des $n$-uplets d'elements disticts de $E$ est transitive.
\end{defn}

\subsection{Groupe quotient}

Soit $(G, \cdot)$ un groupe, on munit l'ensemble $\mathcal{P}(G)$
de la loi de composition interne suivante
\begin{align*}
	\mathcal{P}(G) \times \mathcal{P}(G) &\to \mathcal{P}(G)\\
	(A, B) &\mapsto A \cdot B := \{a \cdot b\ |\ a \in A, b \in B\}.
\end{align*}
Soit $A \in \mathcal{P}(G)$, on définit
\begin{equation*}
	A^{-1} := \{a^{-1}\ |\ a \in A\}.
\end{equation*}

\begin{lem}
	\label{hh-h}
	Soit $G$ un groupe et $H \subset G$ un sous-groupe, alors
	\begin{equation*}
		H^{-1} = H \textrm{ et } H \cdot H = H.
	\end{equation*}
\end{lem}
\begin{proof}
	La première égalité est evidente, car un groupe est invariant par inversion.
	Pour la seconde égalité on a que pour tout $h, h' \in H$, $h \cdot h' \in H$,
	et donc $H \cdot H \subset H$, aussi $H = {e_G} \cdot H \subset H \cdot H$
\end{proof}

% existence-groupe-quotient

\begin{thm}[Existence du groupe quotient]
	\label{existence-quotient}
	Soit $G$ un groupe et $H \triangleleft G$ un sous-groupe distingué.
	Alors le quotient $G/H$ muni de la loi de composition des ensembles
	\begin{align*}
		\mathcal{P}(G) \times \mathcal{P}(G) &\to \mathcal{P}(G)\\
		(A, B) &\mapsto A \cdot B := \{a \cdot b\ |\ a \in A, b \in B\}.
	\end{align*}
	a une structure de groupe. Le groupe $G/H$ ainsi obtenu
	s'appelle le groupe quotient de G par H.
\end{thm}

\begin{proof}
	Nous allons vérifies les quatrues conditions qui font de $(G/H, \cdot)$ un groupe
	\begin{enumerate}
		\item Soient $A, B, C \in G/H$, alors
			\begin{align*}
				A \cdot (B \cdot C) &= A \cdot \{b \cdot c\ |\ b \in B, c \in C\}
					= \{a \cdot (b \cdot c)\ |\ a \in A, b \in B, c \in C\}\\
				&= \{(a \cdot b) \cdot c\ |\ a \in A, b \in B, c \in C\}
					= \{a \cdot b\ |\ a \in A, b \in B\} \cdot C\\
				&= (A \cdot B) \cdot C.
			\end{align*}
			Ainsi on a bien que la loi de composition est associative.
		\item Soient $A, B \in G/H$, alors il existe $g, g' \in G$ tel que $A = g \cdot H$ et
			$B = g' \cdot H$ et on a que	
			\begin{equation*}
				(g \cdot H) \cdot (g' \cdot H)
				= g \cdot (g' \cdot g'^{-1}) \cdot H \cdot g' \cdot H
				= g \cdot g' \cdot \Ad(g'^{-1})(H) \cdot H.
			\end{equation*}
			Et en effet, puisqu'on a fait l'hypothèse que $H \triangleleft G$
			est un sous-groupe distingué,
			on a que ${Ad(x)(H) = H}$ pour tout $x \in G$.
			De plus en invoquant le Lemme \ref{hh-h} on a que $H \cdot H = H$, donc
			\begin{equation*}
				g \cdot g' \cdot \Ad(g'^{-1})(H) \cdot H = g \cdot g' \cdot H \cdot H
				= (g \cdot g') \cdot H \in G/H.
			\end{equation*}
			Donc pour tout $A, B \in G/H$, on a que $A \cdot B \in G/H$.
		\item $H$ est l'element neutre de $(G/H, \cdot)$,
			en effet soit $g \cdot H \in G/H$, alors 
			\begin{align*}
				(g \cdot H) \cdot H &= g \cdot H\\
				H \cdot (g \cdot H) &= g \cdot g^{-1} \cdot H \cdot g \cdot H
				= g \cdot H \cdot H = g \cdot H.
			\end{align*}
		\item Soit $g \cdot H \in G/H$, alors $(g \cdot H)^{-1} = g^{-1} \cdot H \in G/H$,
			en effet
			\begin{equation*}
				g \cdot H \cdot (g \cdot H)^{-1} = g \cdot H \cdot g^{-1} \cdot H
				= H \cdot H = H.
			\end{equation*}
	\end{enumerate}
	Ainsi on a démontré que $(G/H, \cdot)$ est un groupe.
\end{proof}

\begin{prop}
	Soit $G$ un groupe, $H \triangleleft G$ un sous-groupe distingué,
	alors $\red_H : G \to G/H$ est un morphisme surjectif de groupes
	et $\ker(\red_H) = H$.
\end{prop}

\begin{proof}	
	Dans la preuve du Théorème \ref{existence-quotient}, on a montré que pour tous
	$g, g' \in G$, on a que $(g \cdot H) \cdot (g' \cdot H) = (g \cdot g') \cdot H$.
	Ainsi pour tous $g, g' \in G$, il en résulte que
	\begin{equation*}
		\red_H(g) \cdot \red_H(g') = (g \cdot H) \cdot (g' \cdot H)
		= (g \cdot g') \cdot H = \red_H(g \cdot g').
	\end{equation*}
	L'application $red_H$ est surjective, et puisque $H$ est un groupe, $e_G \in H$,
	et donc
	\begin{equation*}
		g \cdot H = H \iff g \in H.	
	\end{equation*}
	Et donc $\ker(\red_H) = H$.
\end{proof}

% theoreme-noyau-image

\begin{thm}[Theorème noyau-image]
	\label{noyau-image}
	Soit $\phi : G \to G'$ un isomorphisme de groupes de noyau $H \triangleleft G$.
	Alors $\phi$ induit l'isomorphisme de groupes
	\begin{align*}
		\phi_H: G/H &\xrightarrow{\sim} \im(\phi)\\
		g H &\mapsto \phi(g).
	\end{align*}
\end{thm}

\begin{proof}
	$\phi_H$ est bien définie, car si pour $g H, g'H \in G/H$, on a $gH = g'H$,
	alors il existe $h \in H$ tel que $g' = gh$ et
	\begin{equation*}
		\phi_H(g'H) = \phi(gh) = \phi(g)\phi(h) = \phi(g) = \phi_H(gH).
	\end{equation*}
	$\phi_H$ est injectif, car pour $gH \in G/H$, on a
	\begin{equation*}
		\phi_H(gH) = e_{G'} \iff \phi(g) = e_{G'} \iff g \in H \iff gH = H = e_{G/H}.
	\end{equation*}
	$\phi_H$ est surjectif, car pour $y \in \im(\phi)$, il existe $g \in G$ 
	tel que $\phi(g) = y$, or $\phi_H(gH) = \phi(g) = y$.
	Enfin, $\phi_H$ est un morphisme de groupes car pour tout $g, g' \in G$,
	\begin{equation*}
		\phi_H(gH \cdot g'H) = \phi_H(gg'H) = \phi(g)\phi(g') = \phi_H(gH)\phi_H(g'H)
	\end{equation*}
\end{proof}

\begin{cor}
	Si $G$ est fini (alors $G'$ aussi car $G \cong G'$), $G/H$ est fini et on a
	$|G/H| = |\im(\phi)|$, car $G/H \cong \im(\phi)$.
	Par Lagrange, $|G| = |\im(\phi)||H|$.
\end{cor}

% propriete-universelle-des-groupes-quotient

\begin{thm}[Propriété universelle des groupes quotient]
	\label{prop-quotient}
	Soient $G, G'$ deux groupes et $H \triangleleft G$ un sous-groupe distingué.
	Alors l'application
	\begin{align*}
		\Psi : \Hom(G/H, G') &\to \{\phi \in \Hom(G, G')\ |\ \ker(\phi) = H\}\\
		\phi_H &\mapsto \phi := \phi_H \circ \red_H
	\end{align*}
	est une bijection.
\end{thm}

\begin{proof}
	Montrons que $\Psi$ est surjective -- soit $\phi \in \Hom(G, G')$
	tel que $\ker(\phi) = H$, soit $\phi_H$ le défini dans le Theorème \ref{noyau-image}. 
	Alors pour tout $g \in G$,
	\begin{equation*}
		\Psi(\phi_H)(g) = (\phi_H \circ \red_H) (g) = \phi_H(gH) = \phi(g)
	\end{equation*}
	Ainsi $\phi \in \im(\Psi)$ et donc $\Psi$ est surjective.  

	Montrons que $\Psi$ est injective -- soient $\phi_H, \phi'_H \in \Hom(G/H, G')$
	tels que $\Psi(\phi_H) = \Psi(\phi'_H)$, alors
	\begin{equation*}
		(\phi_H \circ \red_H)(g) = (\phi'_H \circ \red_H)(g)	
		\iff \phi(gH) = \phi'(gH),
	\end{equation*}
	pour tout $g \in G$. Ainsi $\phi_H = \phi'_H$ et donc $\Psi$ est injective
	et par conséquent aussi bijective.
\end{proof}



\section{Pavages du plan}

\subsection{Tuiles}




\end{document}
