\section{Espace Euclidien et ses Isometries}

\subsection{Espaces Affines}

\begin{definition}
	Soit $G$ un groupe et $X$ un $G$-ensemble. $X$ est un espace principal
	homogène sous l'action de $G$ si
	\begin{enumerate}
		\item $G$ agit transitivement sur $X$
		\item Pour tout $x \in X$, le stabilisateur $G_x$ est trivial
	\end{enumerate}
\end{definition}

\begin{proposition}
	\label{prop:sous-espace-orbite}
	Soit $X$ un espace principal homogène, alors pour tout $\vec{x} \in X$,
	l'application
	\begin{equation*}
		\varphi : g \in G \mapsto g\cdot\vec{x} \in X
	\end{equation*}
	est une bijection.
\end{proposition}

\begin{proof}
	Par le Theorème orbite-stabilisateur on a un isomorphisme de $G$-ensembles
	\begin{equation*}
		G/G_\vec{x} \cong G\cdot\vec{x}
	\end{equation*}
	défini par
	\begin{align*}
		\varphi : G/G_\vec{x} &\xrightarrow{\sim}  G\cdot\vec{x}\\
		gG_\vec{x} &\mapsto g \cdot \vec{x}
	\end{align*}
	et $G\cdot\vec{x} = X$ car l'action est transitive et $G_\vec{x} = \{e_G\}$
	et donc $G/G_\vec{x} = G$. Ainsi $G \cong X$.
\end{proof}

\begin{proposition}
	Soit $X$ un espace principal homogène sous l'action de $G$. Soit $\phi$ le
	morphisme associé à cette action. Alors $\phi$ est injectif.
	De plus $\im(\phi) \cong G$.
\end{proposition}

\begin{proof}
	Puisque pour tout $\vec{x} \in X$, $G_\vec{x} = \{e_G\}$, $\ker(\phi) =
	\{e_G\}$.
\end{proof}

\begin{definition}
	Soit $V$ un $K$-espace vectoriel de dimension finie sur un corps $K$. Un
	espace affine $X$ sous $V$ est un $V$-ensemble (quand on voit $V$ comme le
	groupe additif $(V, +)$) qui est principal homogène. On dit que $V$ est la
	direction de $X$. 
\end{definition}

\begin{notation}
	Le groupe des translations de $X$ sous l'action de $V$ sera noté
	\begin{equation*}
		T(V) = t_V = \{\phi(\vec{v})\ |\ v \in V\} \subset \Bij(X)
	\end{equation*}
	où $\phi$ est le morphisme associé à l'action.
\end{notation}

\begin{notation}
	On notera l'action de $V$ sur $X$ avec un $+$. Vu que $(V, +)$ est
	commutatif, pour $\vec{v} \in V, \vec{x} \in X$, on a que
	$\vec{v} + \vec{x} = \vec{x} + \vec{v}$ (c'est une action à droite ou à
	gauche de manière équivalente).
\end{notation}

\begin{definition}
	Soit $X$ nun espace affine de direction $V$. On défini la dimension de $X$
	comme étant la dimension de $V$:
	\begin{equation*}
		\dim(X) = \dim(V)
	\end{equation*}
\end{definition}

\begin{definition}
	Soit $X$ un espace affine de direction $V$. Un sous-espace affine de $Y
	\subset X$ est un sous-ensemble de $X$ qui est un espace affine de direction
	$W$, où $W$ est un sous-espace vectoriel de $V$.
\end{definition}

\begin{remark}
	Par la Proposition \ref{prop:sous-espace-orbite}, un sous-espace affine est
	de la forme
	\begin{equation*}
		Y = \vec{y_0} + W = \{\vec{y_0} + \vec{w}, \vec{w} \in W\}.
	\end{equation*}
	pour un certain $\vec{y_0} \in Y$. On dit que $Y$ est un sous-espace affine
	de direction $W$ passant par $\vec{y_0}$. On a $\dim(Y) = \dim(W)$. Un
	sous-espace affine de dimension $\dim(Y) = \dim(X) - 1$ est appelé un
	hyperplan affine.
\end{remark}
