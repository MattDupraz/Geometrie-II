\section{Espace Euclidien et ses Isometries}

\subsection{Espaces Affines}

\begin{definition}
	Soit $G$ un groupe et $X$ un $G$-ensemble. $X$ est un espace principal
	homogène sous l'action de $G$ si
	\begin{enumerate}
		\item $G$ agit transitivement sur $X$
		\item Pour tout $x \in X$, le stabilisateur $G_x$ est trivial
	\end{enumerate}
\end{definition}

\begin{proposition}
	\label{prop:sous-espace-orbite}
	Soit $X$ un espace principal homogène, alors pour tout $\vec{x} \in X$,
	l'application
	\begin{equation*}
		\varphi : g \in G \mapsto g\cdot\vec{x} \in X
	\end{equation*}
	est une bijection.
\end{proposition}

\begin{proof}
	Par le Theorème orbite-stabilisateur on a un isomorphisme de $G$-ensembles
	\begin{equation*}
		G/G_\vec{x} \cong G\cdot\vec{x}
	\end{equation*}
	défini par
	\begin{align*}
		\varphi : G/G_\vec{x} &\xrightarrow{\sim}  G\cdot\vec{x}\\
		gG_\vec{x} &\mapsto g \cdot \vec{x}
	\end{align*}
	et $G\cdot\vec{x} = X$ car l'action est transitive et $G_\vec{x} = \{e_G\}$
	et donc $G/G_\vec{x} = G$. Ainsi $G \cong X$.
\end{proof}

\begin{proposition}
	Soit $X$ un espace principal homogène sous l'action de $G$. Soit $\phi$ le
	morphisme associé à cette action. Alors $\phi$ est injectif.  De plus
	$\im(\phi) \cong G$.
\end{proposition}

\begin{proof}
	Puisque pour tout $\vec{x} \in X$, $G_\vec{x} = \{e_G\}$, $\ker(\phi) =
	\{e_G\}$.
\end{proof}

\begin{definition}
	Soit $V$ un $K$-espace vectoriel de dimension finie sur un corps $K$. Un
	espace affine $X$ sous $V$ est un $V$-ensemble (quand on voit $V$ comme le
	groupe additif $(V, +)$) qui est principal homogène. On dit que $V$ est la
	direction de $X$. 
\end{definition}

\begin{notation}
	Le groupe des translations de $X$ sous l'action de $V$ sera noté
	\begin{equation*}
		T(V) = t_V = \{\phi(\vec{v})\ |\ v \in V\} \subset \Bij(X)
	\end{equation*}
	où $\phi$ est le morphisme associé à l'action.
\end{notation}

\begin{notation}
	On notera l'action de $V$ sur $X$ vvec un $+$. Vu que $(V, +)$ est
	commutatif, pour $\vec{v} \in V, \vec{x} \in X$, on a que $\vec{v} + \vec{x}
	= \vec{x} + \vec{v}$ (c'est une action à droite ou à gauche de manière
	équivalente).
\end{notation}

\begin{definition}
	Soit $X$ un espace affine de direction $V$. On défini la dimension de $X$
	comme étant la dimension de $V$:
	\begin{equation*}
		\dim(X) = \dim(V)
	\end{equation*}
\end{definition}

\begin{definition}
	Soit $X$ un espace affine de direction $V$. Un sous-espace affine de $Y
	\subset X$ est un sous-ensemble de $X$ qui est un espace affine de direction
	$W$, où $W$ est un sous-espace vectoriel de $V$.
\end{definition}

\begin{remark}
	Par la Proposition \ref{prop:sous-espace-orbite}, un sous-espace affine est
	de la forme
	\begin{equation*}
		Y = \vec{y_0} + W = \{\vec{y_0} + \vec{w}, \vec{w} \in W\}.
	\end{equation*}
	pour un certain $\vec{y_0} \in Y$. On dit que $Y$ est un sous-espace affine
	de direction $W$ passant par $\vec{y_0}$. On a $\dim(Y) = \dim(W)$. Un
	sous-espace affine de dimension $\dim(Y) = \dim(X) - 1$ est appelé un
	hyperplan affine.
\end{remark}

\subsubsection{Relations de Chasles}

Soit $X$ un espace affine de direction $V$. Soient $P, Q \in X$, alors il existe
un unique vecteur $\vec{v}$ qui envoie $P$ sur $Q$.
\begin{equation*}
	P + \vec{v} = Q
\end{equation*}

\begin{notation}
	On note $\vec{v} = \vvec{PQ} = Q - P$
	On a alors $P + \vvec{PQ} = P + Q - P = Q$
\end{notation}

\begin{proposition}
	Pour tout $P, Q, R \in X$ on a
	\begin{equation*}
		\vvec{PR} = \vvec{PQ} + \vvec{QR}
	\end{equation*}
	En particulier
	\begin{equation*}
		\vvec{PQ} = -\vvec{QP}
	\end{equation*}
\end{proposition}

\begin{proof}
	$\vvec{PR}$ est l'unique vecteur qui envoie $P$ sur $R$. On va montrer que
	$\vvec{PQ} + \vvec{PR}$ envoie $P$ sur $R$. Par transitivité de l'action,
	\begin{equation*}
		P + (\vvec{PQ} + \vvec{QR}) = (P + \vvec{PQ}) + \vvec{QR} = Q + \vvec{QR}
		= R
	\end{equation*}
\end{proof}

\begin{remark}
	Dans un espace affine on ne peut pas additionner des points, mais on peut les
	soustraire.
	\begin{align*}
		X\times X &\to V\\
		(P, Q) &\mapsto Q-P
	\end{align*}
\end{remark}

\begin{proposition}
	Soit $n \geq 1, P_1, \dots, P_n \in X$ des points et
	$\mu_1, \dots, \mu_n \in \K$ des scalaires tels que
	\begin{equation*}
		\sum_{i=1}^{n}\mu_{i} = 0_{\K}
	\end{equation*}
	Alors étant donné $P_0 \in X$, le vecteur
	\begin{equation*}
		\sum_{i = 1}^{n}\mu_i\left(P_i - P_0\right) =
		\sum_{i=1}^n \mu_i \vvec{P_0 P_i} \in V
	\end{equation*}
	ne dépend pas du choix de $P_0$. On notera ce vecteur
	\begin{equation*}
		\sum_{i=1}^n \mu_i P_i
	\end{equation*}
\end{proposition}

\begin{remark}
	Ceci généralise la notation $Q - P$.
\end{remark}

\begin{proof}
	Soit $P'_0 \in X$, alors on veut montrer que
	\begin{equation*}
		\sum_{i=1}^n \mu_i \vvec{P_0 P_i} = \sum_{i=1}^n \mu_i \vvec{P'_0 P_i}
	\end{equation*}
	On calcule la différence
	\begin{equation*}
		\sum_{i=1}^n \left(\mu_i \vvec{P_0 P_i} - \mu_i \vvec{P'_0 P_i}\right) =
		\sum_{i=1}^n \mu_i\left(\vvec{P_0 P_i} - \vvec{P'_0 P_i}\right) = 
		\sum_{i=1}^n \mu_i \vvec{P_0 P'_0} = \vec{0}
	\end{equation*}
\end{proof}

\subsubsection{Barycentres}

\begin{proposition}
	Soit $X$ un espace affine, $n \geq 1, (P_1, \dots, P_n) \in X^n$ un
	$n$-uple de points et $(\lambda_1, \dots, \lambda_n) \in \K^n$ un
	$k$-uple de scalaires verifiant
	\begin{equation*}
		\sum_{i=1}^n \lambda_i = 1_{\K}
	\end{equation*}
	Soit $P_0 \in X$ un point de $X$, alors le vecteur
	$\sum_{i=1}^{n} \lambda_i P_i - P_0 = (-1_K)P_0 + \sum_{i=1}^{n} \lambda_i P_i$
	est bien défini et le point translaté
	\begin{equation*}
		P_0 + \left(\sum_{i=1}^{n}\lambda_i P_i - P_0 \right) \in X
	\end{equation*}
	ne dépend pas du choix de $P_0$. On l'appele le barycentre (algébrique) des
	points $(P_1, \dots, P_n)$ par rapport aux poids $(\lambda_1, \dots, \lambda_n)$ 
	et on le note
	\begin{equation*}
		\Bary(P_1, \dots, P_n; \lambda_1, \dots \lambda_n)	
	\end{equation*}
	Si tous les poids sont égaux ($\lambda_1 = \dots = \lambda_n$ de sorte que
	$\sum_{i} \lambda_i = n \cdot \lambda_1 = 1_\K$), on note ce barycentre
	\begin{equation*}
		\Bary(P_1, \dots, P_n)
	\end{equation*}
\end{proposition}

\begin{proof}
	Soit $P'_0 \in X$ un autre point, on veut montrer que le vecteur qui envoie
	le barycentre construit à partir de $P_0$ sur le barycentre construit à
	partir de $P'_0$ est le vecteur nul.
	\begin{equation*}
		P_0 + \left(\sum_{i=1}^n \lambda_i P_i - P_0\right) -
		\left(P'_0 + \left(\sum_{i=1}^n \lambda_i P_i - P'_0\right)\right)
	\end{equation*}
	et ceci vaut
	\begin{equation*}
		P_0 - P'_0 + \sum_{i=1}^n \lambda_i \left(P_i - P_i\right) + P'_0 - P_0
		= \vvec{P'_0 P_0} + \vec{0} + \vvec{P_0 P'_0} = \vec{0}
	\end{equation*}
\end{proof}

\begin{remark}
	Ainsi on peut former des combinaison linéaires des points dans un espace
	affine si les coefficients de la combinaison linéaire on $1_\K$ pour somme.
	Par exemple si $\Char(\K) \neq 2$, pour $P_1, P_2 \in X$,
	$\frac{1}{2}P_1 + \frac{1}{2}P_2 \in X$.
\end{remark}

\begin{definition}
	Soit $X$ un espace affine de dimension $d$ et de direction $V$; un
	$(d+1)$-uplet de points $P_0, \dots, P_d \in X$ est en position générale si
	les vecteurs $\vvec{P_0 P_1}, \dots, \vvec{P_0 P_d} \in V$ forment une base
	de $V$. On dira que $(P_0, \dots, P_1)$ forme une base affine de $X$.
\end{definition}

\begin{proposition}
	Soeint $X$ un espace affine de dimension $d$ et $P_0, \dots, P_d \in X$ en
	position générale. Pour tout point $P \in X$ il existe un unique
	$(d+1)$-uplet $(\lambda_0, \dots, \lambda_d) \in \K^{d+1}$ tel que
	$\lambda_0 + \dots + \lambda_d = 1$ et
	$P = \Bary(P_0, \dots, P_d; \lambda_0, \dots, \lambda_d)$
\end{proposition}

\begin{proof}
	Laissé au lecteur en exercice.
\end{proof}

\begin{definition}
	Le $(d+1)$-uplet $(\lambda_0, \dots, \lambda_d)$ de la proposition précédente
	sont les coordonnées
	barycentriques de $P$ dans la base affine $(P_0, \dots, P_d)$.
\end{definition}

\begin{definition}
	Une famille affine génératrice est une famille telle que tout point $P \in X$
	peut être obtenu comme barycentre des points de la famille.
\end{definition}

\begin{definition}
	Une famille affine libre est une famille telle que tout barycentre des points
	de cette famille n'est égal à aucun autre barycentre des points de cette
	famille.
\end{definition}

\begin{definition}
	un sous-espace affine $Y \subset X$ est un sous-ensemble de $X$ obtenu comme
	l'ensembe de tous les barycentres possibles de $n+1$ points de $X$ --- 
	$P_0, \dots, P_n \in X$ pour $n \geq 0$ un entier:
	\begin{equation*}
		Y = \Big\{\Bary(P_0, \dots, P_n; \lambda_0, \dots, \lambda_n), 
		\lambda_0, \dots, \lambda_n \in \K, \sum_{i=0}^n \lambda_i = 1\Big\}
	\end{equation*}
	On dira alors que $Y$ est le sous-espace affine engendré par $P_0, \dots, P_n$
	ou encore passant par $P_0, \dots, P_n$.
\end{definition}

\subsubsection{Morphismes d'espaces affines}

\begin{definition}
	Soient $X$ et $Y$ deux epsaces affines (de directions $V$ et $W$). Une
	application $\phi: X \to Y$ est dite affine si elle preserve les barycentres:
	pour tout $(P_1, \dots, P_n) \subset X^n$ et tout
	$(\lambda_1, \dots, \lambda_n) \in \K^n$ verifiant
	$\sum_{i=1}^n \lambda_i = 1$, on a
	\begin{equation*}
		\phi(\Bary(P_1, \dots, P_n; \lambda_1, \dots, \lambda_n)) = 
		\Bary(\phi(P_1), \dots, \phi(P_n); \lambda_1, \dots, \lambda_n)
	\end{equation*}
\end{definition}

\begin{proposition}[Composition d'applications affines]
	Soient $X, Y, Z$ des espaces affines et $\phi: X \to Y$, 
	$\psi: Y \to Z$ des applications affines. Alors
	$\psi \circ \phi : X \to Z$ est affine et sa partie linéaire est
	\begin{equation*}
		(\psi \circ \phi)_0 = \psi_0 \circ \phi_0
	\end{equation*}
	Supposons que $\phi: X \to Y$ soit bijective, alors sa réciproque $\phi^{-1}$
	est affine et $(\phi^{-1}_0) = \phi_0^{-1}$
\end{proposition}

\begin{proof}
	D'une part
	\begin{equation*}
		\psi \circ \phi (P_0 + \vec{v}) = \psi(\phi(P_0) + \phi_0(\vec{v}))
		= \psi(\phi(P_0)) + \psi_0(\phi_0(\vec{v}))
	\end{equation*}
	et d'autre part 
	\begin{equation*}
		\psi \circ \phi (P_0 + \vec{v})
		= \psi \circ \phi(P_0) + (\psi \circ \phi)_0(\vec{v})
	\end{equation*}
	Et donc on a bien que $(\psi \circ \phi)_0 = \psi_0 \circ \phi_0$ 
\end{proof}

\begin{definition}
	\begin{itemize}
		\item Un \emph{endomorphisme} affine est une application affine d'un
			espace affine sur lui-même.
		\item Un isomorphisme d'espaces affines est une application affine
			qui est bijective et dont l'applcation reciproque est encore affine.
		\item Un automorphisme affine est un isomorphisme d'une espace affine sur
			lui-même. On l'appelle également transformation affine.
	\end{itemize}
	On note ses ensebles respectivement par
	\begin{equation*}
		\Hom_{aff}(X, Y) = A \Hom(X, Y), \End_{aff}(X) = A\End(X),
		\Aut_{aff}(X) = A\GL(X)
	\end{equation*}
\end{definition}



\begin{theorem}
	Soient $X, Y$ deux espace affines (de directions $V$ et $W$). Une application
	$\phi: X \to Y$ est affine ssi pour $P_0 \in X$ l'application
	\begin{align*}
		\phi_0: V &\to W\\
		\vec{v} &\mapsto \phi_0(\vec{v}) = \phi(P_0 + \vec{v}) - \phi(P_0)	
	\end{align*}
	est linéaire. Si c'est le cas, alors $\phi_0$ ne dépend pas du choix de
	$P_0$. On l'appelle la partie linéaire de $\phi$ et on a la formule suivante
	\begin{equation*}
		\forall P\in X, \vec{v} \in V, \phi(P+\vec{v}) = \phi(P) + \phi_0(\vec{v})
	\end{equation*}
\end{theorem}

\begin{proof}
	Supposons que $\phi$ soit une application affine. On fixe $P_0 \in X$ et on pose
	\begin{equation*}
		\phi_0(\vec{v}) = \phi(P_0 + \vec{v}) - \phi(P_0)
	\end{equation*}
	On a alors que $\phi_0(\vec{0}) = \phi(P_0) - \phi(P_0) = \vec{0}$. Soient
	$\lambda \in \K, \vec{v}, \vec{u} \in V$, on pose
	$P = P_0 + \vec{u}\textrm{; } Q = P_0 + \vec{v}$.
	Ainsi on a
	\begin{equation*}
		\Bary(P, Q, P_0; 1, \lambda, -\lambda) = P + \lambda Q  -
		\lambda P_0  = P_0 + \vec{u} + \lambda \vec{v}
	\end{equation*}
	Et donc
	\begin{align*}
		\phi_0(\vec{u} + \lambda \vec{v}) &= \phi(P_0 + \vec{u} + \lambda{v})
		- \phi(P_0) = \phi(\Bary(P, Q, P_0, 1, \lambda, -\lambda)) - \phi(P_0) \\
      &= \Bary(\phi(P), \phi(Q), \phi(P_0), 1, \lambda, -\lambda) 
		- \phi(P_0)\\
		&= \phi(P) + \lambda \phi(Q) - \lambda \phi(P_0) - \phi(P_0)
		= \phi_0(\vec{u}) + \lambda\phi_0(\vec{v})
	\end{align*}
	Ainsi $\phi_0$ est bien linéaire. Montrons que $\phi_0$ ne dépend pas du
	choix de $P_0$. Soit $P'_0 \in X$ et posons
	\begin{equation*}
		\phi'_0(\vec{v}) = \phi(P'_0 + \vec{v}) - \phi(P'_0)
	\end{equation*}
	Alors on a que pour tout $\vec{v} \in V$
	\begin{align*}
		\phi'_0(\vec{v}) &= \phi(P_0 + \vvec{P_0 P'_0} + \vec{v})
		- \phi(P_0 + \vvec{P_0 P'_0}) + \phi(P_0) - \phi(P_0) \\
		&= \phi_0(\vvec{P_0 P'_0} + \vec{v}) - \phi_0(\vvec{P_0 P'_0})
		= \phi_0(\vec{v})
	\end{align*}
	Ainsi pour tout $P \in X$
	\begin{equation*}
		\phi(P + \vec{v}) = \phi(P) + \phi_0(\vec{v})
	\end{equation*}

	Supposons à présent que l'application
	\begin{equation*}
		\phi_0(\vec{v}) = \phi(P_0 + \vec{v}) - \phi(P_0)
	\end{equation*}
	est linéaire. Alors
	\begin{align*}
		\phi(\Bary(P_1, \dots, P_n; \lambda_1, \dots \lambda_n))
		&= \phi\left(\sum_{i=1}^n \lambda_i P_i\right)
		= \phi_0\left(\sum_{i=1}^n \lambda_i P_i - P_0\right) + \phi(P_0)\\
		&= \sum_{i=1}^n \lambda_i \phi_0(P_i - P_0) + \phi(P_0)
		= \sum_{i=0}^n \lambda_i \phi(P_i) \\
		&= \Bary(\phi(P_1), \dots, \phi(P_n); \lambda_1, \dots, \lambda_n))
	\end{align*}
\end{proof}

\begin{corollary}
	Soit $V$ un $\K$-espace vectoriel et $\phi$ un endomorphisme affine de $V$
	sur $V$ alors $\phi$ se décompose de manière unique sous la forme
	\begin{equation*}
		\phi = t_{\phi(\vec{0})}\circ\phi_0
	\end{equation*}
	où $\phi_0$ est la partie linéaire.
\end{corollary}

\begin{corollary}
	L'application "partie linéaire"
	$\lin: \phi \in A\GL(X) \mapsto \phi_0 \in \GL(V)$ est un morphisme de
	groupes de noyau correspondant au groupe des translations $T(V)$. En
	particulier $T(V)$ est distingué dans $A\GL(X)$.
\end{corollary}

\begin{proof}
	Par la proposition précédente à $X=Y=Z$, $\lin$ est un morphisme de
	groupes. Comme
	\begin{equation*}
		t_{\vec{u}}(P) = P + \vec{u},
	\end{equation*}
	on a pour tout $\vec{v}\in V$
	\begin{equation*}
		\lin(t_\vec{u})(\vec{v})=\vvec{t_{\vec{u}}(P_0)t_{\vec{u}}(P_0 + \vec{v})}
		= P_0 + \vec{v} + \vec{u} - (P_0 + \vec{u}) = \vec{v} = \id_V(\vec{v})
	\end{equation*}
	Réciproquement si $\lin(\phi) = \id$ on a
	\begin{equation*}
		\phi(P) = \phi(P_0 + P - P_0) = \phi(P_0) + \phi_0(P - P_0) = \phi(P_0) +
		P - P_0 = P + \vvec{P_0\phi(P_0)}
	\end{equation*}
	 et donc
	 \begin{equation*}
		 \phi=t_{\vvec{P_0\phi(P_0)}}
	 \end{equation*}
	 et $\ker(\lin) = T(V).$
\end{proof}

\begin{corollary}
	Soit $\phi: X\to Y$ une application affine. Son image $\im\phi$ est un
	sous-espace affine de $Y$ sous l'action de $\im\phi_0$.
	Pour tout $y \in Y$, la préimage
	\begin{equation*}
		\phi^{-1}[y] = \{ x\in X\ |\ \phi(x) = y\}
	\end{equation*}
	est soit l'ensemble vide (si $y \notin T$) ou un sous-espace affine de $X$
	sous l'action de $ker(\phi_0)$. On a la relation
	\begin{equation*}
		\dim X = \dim V = \dim (\ker \phi_0) + \dim (\im \phi)
	\end{equation*}
\end{corollary}

\begin{corollary}
	Une application affine $\phi:X\to Y$ est
	\begin{itemize}
		\item injective ssi $\phi_0$ l'est
		\item surjective ssi $\phi_0$ l'est
		\item surjective ssi $\dim \im \phi = \dim Y$
		\item un isomorphisme affine ssi $\phi_0$ est bijective
	\end{itemize}
\end{corollary}

\subsection{L'espace euclidien}

\begin{definition}
	Soit $E$ un espace affine de direction $\R^n$. Alors $E$ est dit
	\emph{euclidien} muni d'un produit scalaire.
\end{definition}

\begin{definition}
	Le \emph{produit scalaire} de deux vecteurs $\vec{u} = (u_1, \dots, u_n)$,
	$\vec{v} = (v_1, \dots, v_n)$ est donné par
	\begin{equation*}
		\vec{u} \cdot \vec{v} = \langle \vec{u}, \vec{v} \rangle := 
		u_1v_1 + \dots + u_nv_n
	\end{equation*}
	Le produit scalaire est une forme bilinéaire symétrique définie positive.
\end{definition}

\begin{proposition}
	On déduit de la symétrie et de la bilinéarité du produit scalairs, de
	propriétés dites de \emph{polarisation}:
	\begin{align*}
		\|\vec{u} \pm \vec{v}\|^2 &= \|\vec{u}\|^2 \pm 2 \langle \vec{u},
		\vec{v} \rangle \\
		\langle \vec{u}, \vec{v} \rangle &= \frac{1}{2}(\|\vec{u}\|^2 +
			\|\vec{v}\| - \|\vec{u} - \vec{v}\|^2) \\
		\langle \vec{u}, \vec{v} \rangle &= \frac{1}{4}(\|\vec{u} -
			\vec{v}\|^2 - \|\vec{u} - \vec{v}\|^2)
	\end{align*}
\end{proposition}

\begin{definition}
	La \emph{norme euclidienne} d'un vecteur $\vec{u}$ est donnée
	par 
	\begin{equation*}
		\|\vec{u}\| = \sqrt{\langle \vec{v}, \vec{v} \rangle}
	\end{equation*}
\end{definition}

\begin{definition}
	La \emph{distance euclidienne} entre deux vecteurs $\vec{v}, \vec{u} \in E$
	est donnée par
	\begin{equation*}
		d(\vec{v}, \vec{u}) = \|\vec{v}-\vec{u}\|
	\end{equation*}
\end{definition}

\begin{theorem}
	La norme euclidienne satisfait les propriétés suivantes
	pour tout $\vec{u}, \vec{v} \in E, \lambda \in \R$:
	\begin{itemize}
		\item Séparation des points: pour tout
			\begin{equation*}
				\|\vec{u}\| = 0 \iff \vec{u} = \vec{0}
			\end{equation*}
		\item Inégalité du traingle:
			\begin{equation*}
				\|\vec{u} + \vec{v}\| \leq \|\vec{u}\| + \|\vec{v}\|
			\end{equation*}
		\item Homogénité
			\begin{equation*}
				\|\lambda\vec{u}\| = |\lambda|\|\vec{u}\|
			\end{equation*}
	\end{itemize}
	La distance euclidienne stasifait des propriétés analogues.
\end{theorem}

\begin{proposition}[Inegalité de Cauchy-Schwarz]
	On a 
	\begin{equation*}
		|\langle \vec{u}, \vec{v}\rangle| \leq \|\vec{u}\|\|\vec{v}\|
	\end{equation*}
	avec égalité ssi $\vec{u}, \vec{v}$ sont proportionnels.
\end{proposition}

\begin{definition}
	Deux vecteurs $\vec{u}, \vec{v} \in E$ tels que $\langle \vec{u}, \vec{v}
	\rangle = 0$, sont dit orthogonaux ou perpendiculaires.
\end{definition}

\begin{proposition}
	Soeint $\vec{v}_1, \dots, \vec{v}_m \in E$ des vecteurs deux a deux
	perpendiculaires, alors $\{\vec{v}_1, \dots \vec{v}_m\}$ est une famille
	libre.
\end{proposition}

\begin{definition}
	Soit $(\vec{v}_1, \dots, \vec{v}_m)$ une famille de vecteurs
	perpendiculaires. Alors si tous le svecteurs sont de norme $1$, alors on dit
	que la famille est orthonormée. Si $m = \dim E$, cette famille est une base
	et est appellée une base orthogonale (resp. orthonormale si la famille est
	orthonormée)
\end{definition}

\begin{proposition}
	Soit $(\vec{v}_1, \dots, \vec{v}_n)$ une base orthogonale de $\R^n$. Pour
	tout vecteur $\vec{x} \in \R^n$ la décomposition en combinaison linéaire de
	cette base s'écrit
	\begin{equation*}
		\vec{x} = \lambda_1 \vec{v}_1 + \dots + \lambda_n \vec{v}_n
	\end{equation*}
	avec
	\begin{equation*}
		\lambda_i = \frac{\langle \vec{x}, \vec{v}_i \rangle}{\langle \vec{v}_i, 
		\vec{v}_i \rangle}, \forall i \in \{1, \dots, n\}
	\end{equation*}
\end{proposition}

\begin{theorem}
	Soit $(\vec{u}_1, \dots, \vec{u}_n)$ une base de $\R^n$, il existe une base
	orthonormée $(\vec{v}_1, \dots, \vec{v}_n)$ orthonormée telle que
	\begin{equation*}
		\spn\{ \vec{u}_1, \dots, \vec{u}_i \} = \spn\{ \vec{v}_1,\dots,
		\vec{v}_i\}, \forall i \in \{1, \dots, n\}
	\end{equation*}
\end{theorem}

\subsection{La structure du groupe des isometries}

\begin{definition}
	Une application $\phi: \R^n \to \R^n$ est une isometrie si elle preserve la
	distance euclidienne:
	\begin{equation*}
		\forall P, Q \in \R^n, d(P, Q) = d(\phi(P), \phi(Q))
	\end{equation*}
	On note $\Isom(\R^n)$ l'ensemble des isometries et $\Isom(\R^n)_0$ le
	sous-ensemble des isometries qui fixent le vecteur nul $\vec{0}$.
\end{definition}

\begin{proposition}
	Soit $\phi \in \Isom(\R^n)_0$, alors $\phi$ preserve le produit scalaire des
	vecteurs:
	\begin{equation*}
		\inner{\phi(\vec{u})}{\phi(\vec{v})} = \inner{\vec{u}}{\vec{v}} 
	\end{equation*}
\end{proposition}

\begin{proof}
	Soient $\vec{u} = \vvec{0P}, \vec{v} = \vvec{0Q}$, alors
	\begin{align*}
		\inner{\phi(\vec{u})}{\phi(\vec{v})} &= \frac{1}{2}(\|\phi(\vec{u})\|^2 +	
		\|\phi(\vec{v})\|^2 - \| \phi(\vec{u}) - \phi(\vec{v}) \|^2)\\
		&= \frac{1}{2}(\| \vec{u} \|^2 + \| \vec{v} \|^2 + d(\phi(P), \phi(Q))^2)\\
		&= \frac{1}{2}(\| \vec{u} \|^2 + \| \vec{v} \|^2 + d(P, Q)^2)\\
		&= \frac{1}{2}(\| \vec{u} \|^2 + \| \vec{v} \|^2
			+ \|\vec{u} - \vec{v}\|^2)\\
		&= \inner{\vec{u}}{\vec{v}}
	\end{align*}
\end{proof}

\begin{theorem}
	Les isometries fixant l'origine $\vec{0}$ sont des applications linéaires sur
	$\R^n$. Ces applications sont bijectives.
\end{theorem}

\begin{proof}
	\begin{align*}
		\| \phi(\lambda \vec{u} + \vec{v})
		&- (\lambda \phi(\vec{u}) + \phi(\vec{v}))\|^2 =\\
		&= \inner{\phi(\lambda \vec{u} + \vec{v})
		- \lambda \phi(\vec{u}) - \phi(\vec{v})}{\phi(\lambda \vec{u} + \vec{v})
		- \lambda \phi(\vec{u}) - \phi(\vec{v})}
		\\
		&= \inner{\phi(\lambda \vec{u} + \vec{v})}{\phi(\lambda \vec{u} + \vec{v})}
		+ \lambda^2 \inner{\phi(\vec{u})}{\phi(\vec{u})}
		+ \inner{\phi(\vec{v})}{\phi(\vec{v})}\\
		&\quad- 2\lambda\inner{\phi(\lambda\vec{u} + \vec{v})}{\phi(\vec{u})}
		- 2\inner{\phi(\lambda\vec{u} + \vec{v})}{\phi(\vec{v})}
		+ 2\lambda\inner{\phi(\vec{u})}{\phi(\vec{v})}
		\\
		&= \inner{\lambda \vec{u} + \vec{v}}{\lambda \vec{u} + \vec{v}}
		+ \lambda^2 \inner{\vec{u}}{\vec{u}}
		+ \inner{\vec{v}}{\vec{v}}\\
		&\quad- 2\lambda\inner{\lambda\vec{u} + \vec{v}}{\vec{u}}
		- 2\inner{\lambda\vec{u} + \vec{v}}{\vec{v}}
		+ 2\lambda\inner{\vec{u}}{\vec{v}}
		\\
		&= \inner{0}{0} = 0
	\end{align*}
	$\phi$ est bijective, car elle est injective: soient $P, Q \in R^n$,
	\begin{equation*}
		\phi(P) = \phi(Q) \implies d(\phi(P), \phi(Q)) = 0 \implies
		d(P, Q) = 0 \implies P = Q
	\end{equation*}
	et c'est une application linéaire entre espace vectoriels de même dimension.
\end{proof}

\begin{definition}
	L'ensemble $\Isom(\R^n)_0$ s'appelle l'ensemble des isometries linéaires de
	l'espace. Par opposition un élement général de $\Isom(\R^n)$ sera appelé une
	\emph{isometrie affine}.
\end{definition}

\begin{theorem}
	Une isometrie est une transformation affine. Ainsi les groupes $\Isom(\R^n),
	T(\R^n)$ et $\Isom(\R^n)_0$ sont des sous-groupes du groupe des
	transformations affines $A\GL(\R^n)$. Le groupe $\Isom(\R^n)_0$ est un
	sous-groupe du groupe $\GL(\R^n)$ des applications linéaires inversibles de
	$\R^n$. Le sous-groupe $T(\R^n)$ est distingué dans $\Isom(\R^n)$ et
	$\Isom(\R^n)$ est engendré par ses deux sous-groupes,
	\begin{equation*}
		\Isom(\R^n) = T(\R^n) \circ \Isom(\R^n)_0
	\end{equation*}
	Toute isometrie $\phi$ se décompose de manière unique sous la forme
	\begin{equation*}
		\phi = t_{\phi(0)} \circ \phi_0
	\end{equation*}
	où $\phi_0$ est la partie linéaire de $\phi$.
\end{theorem}

\begin{proof}
	On a que $T(\R^n), \Isom(\R^n)_0 \subset \Isom(\R^n)$, ce sont des
	sous-groupes du groupe des transformations affines.  Soit
	$\phi\in\Isom(\R^n)$. On décompose $\phi$ en une translation et une isometrie
	fixant l'origine fixant $\vec{0}$.  Posons $\phi_0 = t_{-\phi(\vec{0})} \circ
	\phi$, alors $\phi_0$ est une isometrie (c'est la composée de 2 isometries).
	De plus
	\begin{equation*}
		\phi_0(0) = t_{-\phi(0)} \circ \phi(0) = \phi(0) - \phi(0) = 0
	\end{equation*}
	Ainsi $\phi_0$ est une isometrie fixant l'origine, donc elle est linéaire et
	bijective, et ainsi $\phi$ est bijective.
\end{proof}

\subsection{Le groupe des isometries linéaires et des matrices orthogonales}

\begin{theorem}
	Soit $\phi: \R^n \to \R^n$ une application linéaire. Alors $\phi$ est une
	isometrie ssi l'une des conditions suivantes est satisfaite:
	\begin{enumerate}
		\item $\forall \vec{u}, \vec{v}, \inner{\phi(\vec{u})}{\phi(\vec{v})}
			= \inner{\vec{u}}{\vec{v}}$
		\item $\forall \vec{v} \in \R^n, \|\phi(\vec{v})\| = \|\vec{v}\|$
		\item $\phi$ est inversible et $\forall \vec{u}, \vec{v} \in \R^n,
			\inner{\phi(\vec{u})}{\vec{v}} = \inner{\vec{u}}{\phi^{-1}(\vec{v})}$
		\item $\phi$ transforme toute base orthonormée $(\vec{e}_i)_{i\leq n}$ en
			une base orthonormée $(\phi(\vec{e}_i))_{i\leq n}$
	\end{enumerate}
\end{theorem}

\begin{proof}
	On a que les condition 1. et 2. sont équivalentes et charactéristiques des
	isometries. Si $\phi$ preserve le produit scalaire, alors
	\begin{equation*}
		\inner{\phi(\vec{u})}{\vec{v}} =
		\inner{\phi(\vec{u})}{\phi(\phi^{-1}(\vec{v}))} =
		\inner{\vec{u}}{\phi^{-1}(\vec{v})}
	\end{equation*}
	Réciproquement si 3. est vérifiée, alors
	\begin{equation*}
		\inner{\phi(\vec{u})}{\phi(\vec{v})} =
		\inner{\vec{u}}{\phi(\phi^{-1}(\vec{v}))} =
		\inner{\vec{u}}{\vec{v}}
	\end{equation*}
	Soit $\phi$ une isometrie linéaire et $(\vec{e}_i)_{i\leq n}$ une base
	orthonormée, on a pour $i, j \leq n$
	\begin{equation*}
		\inner{\phi(\vec{e}_i)}{\phi(\vec{e}_j)} = \inner{\vec{e}_i}{\vec{e}_j} =
		\delta_{ij}
	\end{equation*}
	ainsi $(\phi(\vec{e}_i))_{i\leq n}$ est orthonormée. Réciproquement, si
	$\phi$ transforme toute base orthonormée en une base orthonormée,
	\begin{align*}
		\|\phi(\vec{v)}\|^2 &= \inner{\phi(\vec{v})}{\phi(\vec{v})} = 
		\inner{\sum_{i=1}^n \lambda_i\phi(\vec{e}_i)}{\sum_{i=1}^n
		\lambda_j\phi(\vec{e}_j)}\\
		&= \sum_{i,j=1}^n\lambda_i\lambda_j\inner{\phi(\vec{e}_i)}{\phi(\vec{e}_j)}
		= \sum_{i=1}^n\lambda_i^2 = \|\vec{v}\|^2
	\end{align*}
\end{proof}

\subsubsection{Matrices des isometries linéaires}

Soit $\phi: \R^n \to \R^n$ une application linéaire. On note $M_\phi$ sa matrice
dans la base canonique:
\begin{equation*}
	M_\phi = (x_{ij})_{1\leq i,j\leq n} =
	\begin{pmatrix}
		x_{11} & \cdots & x_{1n} \\
		\vdots & \ddots & \vdots \\
		x_{n1} & \cdots & x_{nn}
	\end{pmatrix}
\end{equation*}
où pour $1 \leq j \leq n$
\begin{equation*}
	\phi(\vec{e}_j) = \sum_{i=1}^n x_{ij}\vec{e}_i
\end{equation*}

\begin{theorem}
	L'application linéaire $\phi$ est une isometrie ssi $M_{\phi}$ est
	orthogonale.
\end{theorem}

\begin{proof}
	On a que $(\phi(\vec{e}_i))_{i\leq n}$ est une base orthonormale, mais on a
	que $\phi(\vec{e}_i) = (x_{1i}, \cdots, x_{ni})$ exprimé dans la base
	canonique. Mais ceci sont exactement les colonnes de $M_\phi$. Ainsi les
	colonnes de $M_\phi$ forment une base orthonormée et donc $M_\phi$ est
	orthogonale.
\end{proof}

\begin{definition}
	Si $\det M = +1$, alors cette matrice est dite spéciale, et si $\det M = -1$,
	elle est dite non-spéciale. On note respectivement
	\begin{equation*}
		O_n(\R), O_n(\R)^+ = SO_n(\R), O_n(\R)^-
	\end{equation*}
	l'ensemble des matrices orthogonales, orthogonales speciales et orthogonales
	non-spéciales.
\end{definition}

\begin{definition}
	Une isometrie linéaire $\phi$ sera dite spéciale ou non-spéciale si sa
	matrice $M_\phi$ est spéciale ou non-spéciale. On note respectivement
	\begin{equation*}
		\Isom(\R^n)^+_0 = SO(\R^n), \Isom(\R^n)^-_0
	\end{equation*}
	l'ensemble des isometries spéciales ou non-spéciales
\end{definition}

\subsubsection{Orientation}

\begin{proposition}
	Le groupe $\Isom(\R^n)_0$ agit transitivement sur l'ensemble des bases
	orthonormées de $\R^n$ -- $BO_n$:
	soit $B_0 = (\vec{e}_i^0)_{i \leq n}$ la base canonique, on a
	\begin{equation*}
		BO_n = \Isom(\R^n)_0 \cdot B_0
	\end{equation*}
	Le groupe $Isom(\R^n)_0^+$ agit avec deux orbites,
	\begin{align*}
		BO_n^+ &= \Isom(\R^n)_0^+ \cdot B_0 \\
		BO_n^- &= \Isom(\R^n)_0^+ \cdot (-\vec{e}_1^0, \cdots, \vec{e}_n^0)
	\end{align*}
\end{proposition}

\subsubsection{Propriétés spectrales des isometries}

\begin{proposition}
	Soit $\phi$ une isometrie sur $\R^n$, alors toute valeur propre réelle, si
	elle existe, vaut $\pm 1$.
\end{proposition}

\begin{proposition}
	Supposons $n$ impair et soit $\phi$ une isometrie linéaire, alors $\phi$
	possède une valeur propre réelle et un vecteur propre de longueur $1$ -- il
	existe $\vec{e} \in \R^n$ et $\lambda = \pm 1$ avec $\phi(\vec{e}) = \lambda
	\vec{e}$. De plus si $\phi \in SO(\R^n)$, $\lambda = 1$.
\end{proposition}

\begin{proof}
	Le polynôme charactéristique de $\phi$ est d'ordre impair, ainsi il s'anulle
	en un certain point $\lambda$. Par la proposition précédente $\lambda = \pm
	1$. De plus $p_\phi(0) = \det M_\phi$, donc si $\phi \in SO(\R^n)$, on a que 
	$p_\phi(0) = 1$ et comme $\lim_{x\to \infty} p_\phi(x) = -\infty$, par le
	TVI, il existe $\lambda \in ]0, \infty[$ t.q. $p_\phi(\lambda) = 0$, et donc 
	$\lambda = 1$.
\end{proof}

\begin{theorem}
	Soit $\phi \in \Isom(\R^n)_0$ une isometrie linéaire et $W \in \R^n$ un
	sous-espace qui est stable par $\phi$, alors $W^\perp$ est un SEV stable par
	$\phi$. Soit $B$ une base formée d'une BO de $W$ et de $W^\perp$, alors la
	matrice de $\phi$ dans cette base est une matrice blocs:
	\begin{equation*}
		M_{\phi, B} = 
		\begin{pmatrix}
			M_{\phi, B_W} & 0 \\
			0 & M_{\phi, B_{W^\perp}}
			
		\end{pmatrix}
	\end{equation*}
	En particulier on a
	\begin{equation*}
		\det(M_{\phi, B}) = \det(M_{\phi, B_W})\det(M_{\phi, B_{W^\perp}})
	\end{equation*}
\end{theorem}

\begin{proof}
	$W^\perp$ est un SEV de $\R^n$ et $\R^n = W\oplus W^\perp$. Soit $\vec{u} \in
	W^\perp$, on va montrer que $\phi(\vec{u}) \in W^\perp$.  On a que pour tout
	$\vec{w} \in W$, $\inner{\vec{u}}{\vec{w}} = 0$. En particulier, puisque
	$\phi$ est injective, et $\phi(W) \subset W$, ceci implique que $\phi(W) = W$
	et donc pour tout $\vec{w} \in W$, $\phi^{-1}(\vec{w}) \in W$, donc
	$\inner{\vec{u}}{\phi^{-1}(\vec{w})} = 0$, et par conséquent,
	$\inner{\phi(\vec{u})}{\vec{w}} = 0$ pour tout $\vec{u} \in W^\perp$, ainsi
	$W^\perp$ est stable par $\phi$.  La matrice de l'application $\phi$
	restreinte à $W$ et $W'$, est orthogonale dans des BON de $W$ et $W^\perp$
	resp. Si on a deux telles bases, $B_W$ et $B_{W^\perp}$, alors leur union $B$
	est une BON de $\R^n$. Comme $\phi(W) \subset W$ et $\phi(W^\perp) \subset
	W^\perp$, la matrice de $\phi$ dans la base $B$ est une matrice bloc formée
	des deux matrices de $\phi$ associées.
\end{proof}

\begin{theorem}
	Soit $n \geq 1$ un entier impair et $\phi \in \Isom(\R^n)_0$ une isometrie
	linéaire; il existe une base orthonormée que l'on peut supposer orientée
	$B = (\vec{e}_1, \cdots, \vec{e}_n)$ telle que $M_{\phi, B}$ est de la forme
	\begin{equation*}
		M_{\phi, B} =
		\begin{pmatrix}
			\pm 1 & 0 \\
			0 & M_{n-1}
		\end{pmatrix}
	\end{equation*}
	et $\det(M_{\phi, B}) = (\pm 1) M_{n-1}$
\end{theorem}

\begin{proof}
	Puisque $n$ est impair, $\phi$ admet une valeur propre $\vec{v}$ qui associée
	à la valeur propre $\lambda = \pm 1$. L'espace engendré par $\vec{v}$
	est stable par $\phi$, ainsi on conclut par le theorème précédent.	
\end{proof}

\begin{definition}
	L'enseble des isometries
	$\Isom(\R^n)^+ = \{\phi \in \Isom(\R^n), \phi_0 \in \Isom(\R^n)_0^+\}$
	est appellé le groupe des déplacemments affines.
	Son complémentaire est appellé l'ensemble des anti-déplacements affines.
\end{definition}

\begin{theorem}
	L'ensemble des déplacements affines forme un sous-groupe distingué de 
	$\Isom(\R^n)$ d'indice 2.
\end{theorem}

\begin{proof}
	On a que $\Isom(\R^n)^+ = \ker(\det \circ \lin)$, ainsi $\Isom(\R^n)^+$
	est un sousgroupe distingué de $\Isom(\R^n)$.
	l'image de $\det \circ \lin$ est $\{ \pm 1 \}$, ainsi, par le theorème noyau
	image, $\Isom(\R^n)/\Isom(\R^n)^+ \cong \{ \pm 1\}$ et donc l'indice de
	$\Isom(\R^n)^+$ dans $\Isom(\R^n)$ est $2$.
\end{proof}

\begin{proposition}
	Soit $\phi_0$ une isometrie linéaire et $\vec{u} \in \R^n$. L'isometrie
	$\phi = t_{\vec{u}} \circ \phi_0$ admet un point fixe ssi $\vec{u} \in
	\im(\phi_0 - \id)$ et alors l'ensemble des points fixes est un espace affine
	de direction $\ker(\phi_0 - \id)$.
\end{proposition}

\begin{proposition}
	Soit $\phi_0$ une isometrie linéaire, ona a la décomposition en somme directe
	orthogonale
	\begin{equation*}
		\R^n = \ker(\phi_0 - \id) \oplus^\perp \im(\phi_0 - \id)
	\end{equation*}
\end{proposition}

\begin{proof}
	Soit $\vec{v} \in \ker(\phi_0 - \id)$ et $\vec{w} \in \im(\phi_0 - \id)$,
	alors puisque $\phi_0(\vec{v}) = \vec{v}$ et il existe $\vec{z}$ t.q.
	$\phi_0(\vec{z}) - \vec{z} =\vec{w}$, on a
	\begin{equation*}
		\inner{\vec{v}}{\vec{w}} = \inner{\phi_0(\vec{v})}{\phi_0(\vec{z})}
		- \inner{\vec{v}}{\vec{z}} = 0
	\end{equation*}
	Ainsi $\ker(\phi_0 - \id)$ est perpendiculaire à $\im(\phi_0 - \id)$ et comme
	par le theorème du rang
	$\dim(\ker(\phi_0 - \id)) + \dim(\ker(\phi_0 - \id)) = n$, on a bien que
	\begin{equation*}
		\R^n = \ker(\phi_0 - \id) \oplus^\perp \ker(\phi_0 - \id)
	\end{equation*}
\end{proof}

\begin{proposition}
	Soit $\phi$ affine, alors $t_{\vec{v}}$ commute avec $\phi$ ssi $\vec{v} \in
	\ker(\phi_0 - \id)$.
\end{proposition}

\begin{proof}	
	On a que $t_{\vec{v}}$ commute avec $\phi$ ssi elle commute avec $\phi_0$,
	car $\phi = t_{\vec{u}} \circ \phi_0$ et les translations commutent.
	On a que $t_{\vec{v}}\circ \phi_0 \circ t_{\vec{-v}}$ est une application
	affine de la même partie linéaire que $\phi_0$, ainsi elle est égale à
	$\phi_0$ si il y a égalité en un seul point. En particulier si
	\begin{equation*}
		t_{\vec{v}} \circ \phi_0 \circ t_{\vec{-v}} (\vec{v}) = \phi_0(\vec{v})	
		\iff \phi_0(\vec{v}) = \vec{v} \iff \vec{v} \in \ker(\phi_0 - \id)
	\end{equation*}
\end{proof}

\subsection{Classification et nomenclature quand $n=3$}

\subsubsection{Rotations}
Si $\phi \in \Isom(\R^3)_0$ est une matrice speciale, il existe une base $B =
(\vec{e}_1, \vec{e}_2, \vec{e}_3)$ de
$\R^3$ telle que
\begin{equation*}
	M_{\phi, B} = 
	\begin{pmatrix}
		1 & 0 & 0 \\
		0 & c & -s \\
		0 & s & c
	\end{pmatrix}
	\in SO_3(\R)\textrm{ avec }M_2 = 
	\begin{pmatrix}
		c & -s \\
		s & c
	\end{pmatrix}
	\in SO_2(\R)
\end{equation*}
On a alors que $c^2 + s^2 = 1, c = \cos(\theta), s = \sin(\theta)$.
On dit que $\phi$ c'est une \emph{rotation} d'axe $\R\vec{e}_1$ et d'angle
$\theta$, ou $c + is$ en paramètre complexe. L'ensemble des points fixes de
$\phi$ est la
droite $\R\vec{e}_1$.
\begin{remark}
	La base doit être orientée, sinon l'angle peut changer de signe. Mais la
	plupart du temp on s'interessera à l'angle de la rotation au signe près.
\end{remark}

\subsubsection{Symétrie orthogonale}

Soit $\phi \in \Isom(\R^3)_0^-$ une isometrie non-speciale, supposons que la
matrice de $\phi$ soit de la forme
\begin{equation*}
	M_{\phi, B} = 
	\begin{pmatrix}
		1 & 0 & 0 \\
		0 & a & b \\
		0 & c & d
	\end{pmatrix}
	\in O_3(\R)^-
	\textrm{ avec }
	M_2 = 
	\begin{pmatrix}
		a & b \\
		c & d
	\end{pmatrix}
	\in O_2(\R)^-
\end{equation*}
C'est une symétrie dans le plan orthogonal $\vec{e}_1^\perp = \R\vec{e}_2 +
\R\vec{e}_3$. Soit $\vec{e}_2'$ un vecteur appartenant à l'axe de cette
symétrie, et $\vec{e}_3'$ un vecteur unitaire perpendiculaire à $\vec{e}_1$ et
$\vec{e}_2'$. Alors a on que pour $B' = (\vec{e}_1, \vec{e}_2', \vec{e}_3')$
\begin{equation*}
	M_{\phi, B'} =
	\begin{pmatrix}
		1 & 0 & 0 \\
		0 & 1 & 0 \\
		0 & 0 & -1
	\end{pmatrix}
\end{equation*}
dans ce cas c'est une symétrie orthogonale par rapport au plan
$\vec{e}_3^\perp$.

\subsubsection{Anti-rotation}

Le dernier cas est celui d'une matrice de la forme
\begin{equation*}
	M_{\phi, B} =
	\begin{pmatrix}
		-1 & 0 & 0 \\
		0 & c & -s \\
		0 & s & c
	\end{pmatrix}
	\in O_3(\R)^-\textrm{ avec }M_2 = 
	\begin{pmatrix}
		c & -s \\
		s & c
	\end{pmatrix}
\in SO_2(\R)
\end{equation*}
avec $c = \cos(\theta)$ et $s = \sin(\theta)$. Alors $\phi$ performe une
rotation dans le plan orthogonal $\vec{e}_1^\perp$ combinée avec une symétrie
par rapport au plan $\vec{e}_1^\perp$.

\begin{remark}
	Les anti-rotations d'angle $0$ sont exactement les symétries par rapport à
	un plan.
\end{remark}

\begin{remark}
	Une anti-rotation d'angle $\pi$ est l'application linéaire $-\id$ qui est
	une symétrie par rapport au centre $\vec{0}$.
\end{remark}

\subsubsection{Isometries affines}

Considérons une isometrie affine $\phi \in \Isom(\R^3)$, alors $\phi$ se
décompose sous la forme
\begin{equation*}
	\phi = t_\vec{u} \circ t_\vec{v} \circ \phi_0
\end{equation*}
avec
\begin{equation*}
	\vec{u} + \vec{v} = \phi(\vec{0}),
	\vec{u} \in \ker(\phi_0 - \id),
	\vec{v} \in \im(\phi_0 - \id)
\end{equation*}

\subsubsection{Déplacements affines}
\begin{enumerate}
	\item Si $\phi_0 = \id$, $\phi$ est une translation
	\item Si $\phi_0$ est une rotation non-triviale autour de l'axe
		$\ker(\phi_0 - \id)$ et $\vec{u} = \vec{0}$, alors l'ensemble des points
		fixes de $\phi$ est une droite de même direction que
		$\ker(\phi_0 - \id)$. On dit que $\phi$ est une \emph{rotation affine}.
		Si la rotation est d'angle $\pi$, c'est une 
		symétrie orthogonale par rapport à l'axe $\ker(\phi_0 - \id)$.
	\item Si $\phi_0$ est une rotation non-triviale autour de l'axe
		$\ker(\phi_0 - \id)$ et $\vec{u} \neq \vec{0}$, alors $\phi$ n'admet pas
		de points fixes. C'est une composée de rotation affine et translation le
		long de l'axe $\R\vec{u}$. On appelle $\phi$ un \emph{vissage}.
\end{enumerate}

\subsubsection{Antidéplacements affines}
\begin{enumerate}
	\item Si $\phi_0 = -\id$, alors $\phi$ admet un unique point fixe et c'est
		la \emph{reflexion ou symétrie centrale} par rapport à ce point.
	\item Si $\phi_0$ est une symétrie par rapport au plan $\ker(\phi_0 - \id)$
		et $\vec{u} = \vec{0}$, alors $\phi$ est une \emph{réflexion ou une
		symétrie} par rapport à un plan parallèle à $\ker(\phi_0 - \id)$.
	\item Si $\phi_0$ est une symétrie par rapport au plan $\ker(\phi_0 - \id)$
		et $\vec{u} \neq \vec{0}$, alors $\phi$ n'admet pas de points fixes, 
		c'est la composée d'une symétrie et translation de vecteur $\vec{u}$. On
		appelle $\phi$ une \emph{symétrie plane glissée}.
	\item Si $\phi_0$ est une anti-rotation d'axe $\ker(\phi_0 + \id)$ et
		d'angle complexe $\neq 1 ou 0$, on a $\ker(\phi_0 - \id) = \{\vec{0}\}$
		et donc forcément $\vec{u} = \vec{0}$. Ainsi $\phi$ a exactement un point
		fixe. $\phi$ est appellée \emph{anti-rotation} autour de l'axe affine
		passant par son point fixe et de direction $\ker(\phi_0 + \id)$
\end{enumerate}
